
\chapter{Differences between bushfire and heatwave weather patterns in Victoria, Australia}
\newpage{}

\section{Introduction}

Bushfires occur in environmental conditions that also exist during
heatwaves in Victoria, Australia. Hot and dry weather characterises
both type of events. Therefore, a logical train of thought suggests
that bushfires happen during or in the latter stages of heatwaves.
However, this study shows that their co-occurrence is not common.
Chapter 4 describes the typical weather patterns of bushfire and heatwave
in this region. This investigation is important because Victoria frequently
endures these extreme events. Forecasting efforts can benefit from
a deeper understanding of their synoptic differences. The investigation
computed anomaly composites for the period 1961-2011 to compare bushfire
and heatwave weather patterns.

This chapter comprises four sections. Section two describes the data
and methods of this research. Subsequently, the study presents a comparison
between bushfire and heatwave weather patterns. The last section summaries the chapter.


\section{Data and methodology\label{sec:Data-and-methodology}}


\subsection{Data}


\subsubsection{Bushfire activity records}

This thesis analysed two bushfire datasets. The first database belongs
to the Department of Environment, Land, Water and Planning (DELWP)
of the State Government of Victoria. This dataset provides a compilation
of bushfires since 1900 to present. Additionally, it provides the
area burnt by each bushfire event. DELWP's database provides official
information. Therefore, it was selected as the main bushfire data
source for the analyses in this chapter. However, the methodology
for its compilation has not been published or peer-reviewed. Thus,
the use of an alternative bushfire dataset that covers the same period
was necessary to validate this information.

Researchers at Risk Frontiers (Macquarie University) developed a natural
hazard occurrence database (\textquoteleft PerilAUS\textquoteright ).
This dataset comprises events such as tropical cyclones, floods and
bushfires in Australia. The information spans more than two hundred
years to the present \citep{Haynes2010}. This database compiles information
from the Sydney Gazette and Sydney Morning Herald newspapers \citep{Coates1996}.
These newspapers started publishing in 1803 and 1842 respectively.
The information was validated with local newspapers and official information
where necessary \citep{Crompton2010}. According to \citet{Coates1996},
the bushfire subset of this database was initially developed in 1991
with information from the Commonwealth Scientific and Industrial Research
Organisation (CSIRO). 

This database establishes two types of bushfire data: reports and
events (M. Mason, personal communication, August 16, 2013). A registered
bushfire report refers to damage in a particular location. On the
other hand, a bushfire event refers to the date when bushfires occurred\textemdash or
started\textemdash , causing damage in several places. For example,
the 'Black Saturday' fires produced reports in different locations
of Victoria (e.g. St. Andrews, Weerite, Marysville). However, this
case is catalogued as a single bushfire event. 

Risk Frontiers database has limitations. The compilation of records
from newspapers brings biases such as representing only the "newsworthy" events 
(e.g. bushfires that caused deaths and economic impact), while not reporting major events in remote areas. 
Therefore, the frequency of bushfires from this database is a reflection of human fatalities and building
damage rather than fire intensity (K. Haynes, personal communication,
August 13, 2013).


\subsubsection{Climate data \label{sub:Climate-data}}

This investigation used two types of weather and climate data. Reanalysis datasets
and high quality climate observations. This sub-section briefly explains
what reanalysis data are and describes the three datasets chosen:
``The Twentieth Century Reanalysis Project'' (20CR), the Met Office
Hadley Centre\textquoteright s sea ice and sea surface temperature
(SST) dataset (HadISST1), and the Australian Climate Observations
Reference Network (ACORN-SAT). The first two are reanalysis datasets
while the last comprises weather station data. The main criteria to
select these data was the length of its records. The three datasets
cover more than 100 years.

Climate research relies on long-term datasets for understanding climate
variability and change. However, the availability of weather station 
information is often limited. Reanalysis projects
or \textquotedblleft retrospective\textquotedblright{} analysis are
international efforts created to address this problem \citep*{Compo2011}.
These projects produce reliable global circulation datasets. Several
reanalysis projects have been created to date (e.g. the 1948\textendash present
NCEP\textendash NCAR reanalysis dataset, the 1957\textendash 2002
ECMWF reanalysis, the 1979\textendash present Japan Meteorological
Agency reanalysis). 

The Twentieth Century Reanalysis Project (20CR) is a reanalysis product
derived assmilating only sub-daily surface pressure reports,
monthly sea surface temperature and observed sea-ice data as boundary
conditions \citep*{Compo2011}.The 20CR uses an Ensemble Kalman Filter as 
the data assimilation system \citep{Whitaker2011}, a global numerical weather prediction model 
(Global Forecast System (GFS) from NCEP) and sea surface temperature and sea ice data from
the Hadley Centre Sea Ice and SST dataset \citep{Rayner2003}. Version two of the 20CR\textemdash available
when the project started\textemdash yields atmospheric fields for
the period 1871\textendash 2008. This project comprises 56 possible
realisations of the atmosphere. Since in this project it is not feasible
to use every realization, the study analysed the ensemble mean. The
dataset yields 6-hourly daily data with a spatial grid of 2x2\textsuperscript{o}.
The 20CR data is available at 24 vertical levels. Most analyses in
this investigation used the 1000 hPa (near surface) values.

The Met Office Hadley Centre\textquoteright s sea ice and sea surface
temperature (SST) dataset (HadISST1) dataset was generated using a
reduced space optimal interpolation technique to produce high-quality
gridded observations \citep{Rayner2003}. HadISST1 comprises monthly
SST data with a 1x1\textsuperscript{o} resolution. These data spans
the period 1871-present.

These datasets have common intrinsic limitations. For example, a limited
number of observations in the early decades of the 20\textsuperscript{th}
century, particularly in the Southern Hemisphere.
Additionally, not all variables are equally well represented. However,
recent results have shown that the 20CR represents well the inter-annual variations 
of rainfall and temperature in southeastern Australia\textquoteright s climate \citep{Ashcroft2014}. 

On the other hand, the calculation of heatwave events used the Australian
Climate Observations Reference Network (ACORN-SAT) dataset \citep{Trewin2013}.
The Australian Bureau of Meteorology (BoM) generated this high-quality
station dataset (see http://www.bom.gov.au/climate/change/acorn-sat/,
accessed June 7\textsuperscript{th} 2016). Achieving this high-quality
required to homogenise the data. This means that changes in observations
caused by artificial reasons (e.g. weather station site moves, technology
development, random errors, etc.) have been corrected. This dataset
includes daily maximum and minimum surface air temperature values over 112
locations. These variables are required in several methods to define
heatwaves. The weather stations selected for the analyses in this
study were Melbourne Regional Office, Mildura, Nhill and Sale (see
Figure \ref{fig:Location of meterological stations used to compute heatwaves dates in Victoria, Australia}
and Appendix B). These stations provide the longest set of records. 

\begin{figure}[h]
\noindent \begin{centering}
\includegraphics[scale=0.35]{Chapter_4/Figures/Location_ws}
\par\end{centering}

\caption[Location of meterological stations used to compute heatwaves dates
in Victoria, Australia]{Location of meterological stations used to compute heatwaves dates
in Victoria, Australia. \label{fig:Location of meterological stations used to compute heatwaves dates in Victoria, Australia} }


\end{figure}



\subsection{Methodology}


\subsubsection{Analysis of bushfire activity records}

Bushfire records are the main data for this research. The two databases
used have information starting from the early 20\textsuperscript{th}
century. However, their bushfire definitions are different. Therefore,
it was necessary to investigate area-burnt thresholds in DELWP's dataset.
This process allowed comparisons with Risk Frontier's
(RF) database. Fire events that burnt at least the area set by an
arbitrary threshold were aggregated. 

The analyses examined two periods: 1911-1960 and 1961-2011. It started
from 1911 since meteorological data in Victoria is more reliable from
this year. The year 2011 was selected since RF's data was available
until this year. On the other hand, splitting in half the period of
analysis was an important decision. \citet{Wang2013} argue that the
response of climate modes variability can change over long periods
of time. Therefore, this chapter aimed to explore the changes in fire
activity between the first and second half of the 20\textsuperscript{th}
century. The next chapter investigates how these changes are linked
to climate modes variability.

The investigation of fire activity focused on the season December-January-February
(DJF). An appraisal of the datasets revealed that most bushfires in
Victoria occur in DJF. For example, 73\% of bushfires (31 out of 42
events) in DELWP database occurred in this season (1961-2010). For
RF's database, the percentage increases to 83\% (43 out 52 events).


\subsubsection{Computation of heatwave dates}

\citet{Nairn2009} proposed a system to define heatwaves\textemdash with
further details provided in \citet{Nairn2013}\textemdash . The defintion
was adopted because it combines two useful concepts. The first concept
is ``Excess Heat''. This criterion refers to a high heat period
that was not dispersed due to elevated temperatures overnight. The
second concept is ``Heat Stress''. It relates to a period that has
been warmer in comparison to the previous month. Using these propositions,
\citet{Nairn2009} created an ``Excess Heat Factor'':

\begin{equation}
EHI_{sig}=(T_{i}+T_{i+1}+T_{i+2})/3-T_{95}
\end{equation}


\begin{equation}
EHI_{accl}=(T_{i}+T_{i+1}+T_{i+2})/3-(T_{i-1}+...+T_{i-30})/30
\end{equation}


\begin{equation}
EHF=EHI_{sig}*\left|EHI_{accl}\right|
\end{equation}


Where:

$T:$ Daily average of maximum and minimum temperature.

$T_{95}:$ 95\textsuperscript{th} percentile of the long-term daily
temperature (maximum and minimum temperature average).

$EHI_{sig}:$ Excess heat sub-index.

$EHI_{accl}:$ Heat stress sub-index.

$EHF:$Excess heat factor.

The analysis also used a second heatwave definition. \citet{Pezza2012}
describe the alternative approach used. This definition was selected
because it represents extreme heatwave events. \citet{Pezza2012}
define a heatwave as a period of at least 3 consecutive days with
maximum temperature above the 90\textsuperscript{th} percentile.
The criteria also requires that minimum temperature is above the 90\textsuperscript{th}
percentile in the second and third days. The definition uses monthly
climatologies. 


\subsubsection{Bushfire and heatwave composite analysis}

A composite analysis was undertaken to depict the general (average)
prevailing conditions of Victoria during heatwaves and bushfires. 
Daily anomaly composites were calculated from day -5 to day +5 of
each bushfire and heatwave event. The plots only show days -1 to +1 because they better illustrate the
contrasts. These plots describe the dynamical evolution of each phenomenon.
It is important to emphasise that bushfire and heatwave events were
analysed separately. Therefore, events that had coincident dates were
not considered in the calculations. This consideration comprised the
three-heatwave dates definition plus three subsequent days. This window
of days includes bushfires that might have occurred immediately after
a heatwave (see Appendix A for a complete list of bushfire and heatwave
dates). In this chapter, these exclusions were performed comparing
bushfire events from the DELWP database with heatwaves calculated
with \citet{Nairn2009} methodology. 

Finally, seasonal anomaly composites were calculated for austral spring
and summer. These composites represent the large-scale patterns in
the preceding (September-October-November) and concurrent (December-January-February)
bushfire and heatwave seasons. The calculation of a Student t-test
using the computed anomalies tested with a 5\% level the statistical
significance of the results. Chapter 6 replicates these analyses using
Risk Frontiers bushfire database and \citet{Pezza2012} heatwave definition. 


\section{Results and discussion}


\subsection{Bushfire and heatwaves frequency in Victoria}

Table \ref{tab:Bushfire events per area burnt in Victoria} compares
the two bushfire datasets available. DELWP's records present a clear
imbalance between the two periods analysed. This discrepancy is
independent of the area burnt threshold used. In contrast, RF's database
shows approximately equivalent numbers.

On the other hand, DELWP's database registers 31 events that burnt
at least 10,000 Ha. These events occurred during the period 1961-2011.
For the same period, RF database registers 43 events. Therefore, the
analyses used this period. Consequently, this thesis defines a bushfire
as any event that burnt more than 10,000 Ha. Figure \ref{fig:Time series of bushfires, heatwaves and Southern Oscillation Index (SOI) in Victoria, Australia during the period 1961-2011}
shows that the two bushfire databases are moderately correlated (r=0.44,
p=0.00) using this threshold after 1961.

\begin{table}[h]
\caption[Bushfire events per area burnt in Victoria]{Bushfire events per area burnt in Victoria during the season December-January-February
(DJF). Data source: Department of Environment, Land, Water and Planning
(DELWP) and Risk Frontiers (RF). \label{tab:Bushfire events per area burnt in Victoria}}


\noindent \centering{}\noindent\resizebox{\textwidth}{!}{%%
\begin{tabular}{|c|c|c|c|c|c|c|c|c|}
\hline 
\multicolumn{8}{|c|}{DELWP } & RF\tabularnewline
\hline 
\hline 
Period/Burnt area (Ha) & >100 & >300 & >500 & >1000 & >2000 & >5000 & >10000 & No area burnt data available\tabularnewline
\hline 
1911-1960 & 12 & 10 & 7 & 7 & 7 & 6 & \textbf{4} & \textbf{49}\tabularnewline
\hline 
1961-2011 & 338 & 254 & 210 & 161 & 111 & 77 & \textbf{31} & \textbf{43}\tabularnewline
\hline 
\end{tabular}}
\end{table}


Figure \ref{tab:Bushfire events per area burnt in Victoria} also
shows times series of heatwave events in Victoria during the period
1961-2010. The figure illustrates the frequency of events using \citet{Nairn2009}
and \citet{Pezza2012} definitions. The analysis using the 55\textsuperscript{th} percentile
of values computed with \citet{Nairn2009} definition yielded 13 heatwave
events. The computation using \citet{Pezza2012} criteria estimated
a similar number of heatwaves. However, their time series are not correlated (r=0.06, p=0.68).
This lack of correlation reveals that the heatwave definitions used are very sensitive and 
each describes different episodes of heat anomalies.  

Both definitions show that Victoria
can experience one or two severe events during extreme heatwave seasons. 

\begin{figure}[h]
\noindent \begin{centering}
\includegraphics[scale=0.7]{Chapter_4/Figures/BF_HW_time_series}
\par\end{centering}

\caption[Time series of bushfire and heatwave events in Victoria for the period
1961-2011]{Time series of bushfire and heatwave events in Victoria for the period
1961-2011. Panels a) and b) show the bushfire and heatwave events
evolution respectively. For bushfire events, panel a) shows data from
the Department of Environment, Land, Water and Planning (DELWP) (red
solid line) and from Risk Frontiers (RF) (black dashed line). For
heatwave events, panel b) displays red solid and black dashed lines
that correspond to heatwaves computed with \citet{Nairn2009} and
\citet{Pezza2012} definitions respectively. The bushfire time series
have a higher correlation than the heatwaves counterpart (Bushfires:
r=0.44, p=0.00; Heatwaves: r=0.06, p=0.68). \label{fig:Time series of bushfire and heatwave events in Victoria for the period 1961-2011}}
\end{figure}


In this chapter, the investigation used DELWP bushfire database and
heatwaves computed with \citet{Nairn2009} definition. These data
reveals that only 13\% of bushfires (4 events) occurred simultaneously
with heatwaves in Victoria, considering a window of 10 days centred on the day of the event. 
In this case, only one heatwave produced
the fires. These bushfires are linked to ``Black Saturday'' (see
Appendix A for dates). These common days, have been excluded from
the composites. The exclusion yields a ``pure'' signal of bushfires
and heatwaves synoptic weather patterns. 

Figure \ref{fig:Time series of bushfires, heatwaves and Southern Oscillation Index (SOI) in Victoria, Australia during the period 1961-2011}
shows a comparison between bushfire (panel a) and heatwave (panel
b) frequencies. In this figure, a times series of the Southern Oscillation
Index (SOI) has been included. The frequency of bushfire events displays
a discernible climate signal. The polynomial curves show that there
is a clear inter-decadal link between bushfires and the SOI. However,
the inter-annual variability shows only a weak correlation. The R
coefficients between the SOI with bushfires and heatwaves are -0.16
and 0.22 respectively. 

\begin{figure}[h]
\noindent \begin{centering}
\includegraphics[scale=0.7]{Chapter_4/Figures/BF_HW_frecuencies}
\par\end{centering}

\caption[Time series of bushfires, heatwaves and Southern Oscillation Index
(SOI) in Victoria, Australia during the period 1961-2011]{Time series of bushfires, heatwaves and Southern Oscillation Index
(SOI) in Victoria, Australia during the period 1961-2011. Panel a)
shows 27 bushfire events and b) 13 heatwaves (red bars representing
events for the season December-January-February). SOI time series
are presented as the average for the season December-January-February
(blue lines). Monthly SOI values were taken from the Bureau of Meteorology
web page. The bushfire, heatwave and SOI time series were fitted to
a 6\protect\textsuperscript{th} order polynomial curve (black lines).
R coefficients of bushfire and heatwave events with the SOI are -0.16
and 0.22 respectively. Bushfire events data were obtained from the
Department of Environment, Land, Water and Planning (DELWP) database.
Heatwaves were defined as the 55\protect\textsuperscript{th} percentile
of events computed using \citet{Nairn2009} heatwave definition. \label{fig:Time series of bushfires, heatwaves and Southern Oscillation Index (SOI) in Victoria, Australia during the period 1961-2011}}


\end{figure}



\subsection{Weather patterns for daily climatology }

The air temperature (1000 hPa level) anomalies show differences between
bushfires and heatwaves events (see Figure \ref{fig: Air temperature daily anomalies at 1000 hPa level during extreme bushfire and heatwave events in Victoria, Australia during the period 1961-2010}).
Although the patterns are similar, the anomalies are weaker in the
bushfire composites. This includes the warm area in southeastern Australia.
The pattern also shows negative anomalies observed in Queensland and
Western Australia. Bushfires show a cold front on day zero. Frontal activity is associated with strong winds that trigger extreme bushfire events. Fire weather
temperature anomalies over Victoria (+ 3 \textsuperscript{o}C) are
weaker than heatwave anomalies (+ 5 \textsuperscript{o}C). In contrast,
the dynamics of heatwave weather is more static than its bushfire
counterpart. This means that although the patterns are similar, the
heatwave anomalies remain in the same position longer that the bushfire
anomalies. 

\begin{figure}[h]
\noindent \begin{centering}
\includegraphics[scale=0.65]{Chapter_4/Figures/T_daily}
\par\end{centering}

\caption[Air temperature daily anomalies at 1000 hPa level during extreme bushfire
and heatwave events in Victoria, Australia during the period 1961-2010]{Air temperature daily anomalies at 1000 hPa level during extreme bushfire
and heatwave events in Victoria, Australia during the period 1961-2010.
Panels a), b), c), d), e) and f) show anomalies for day -1 to day
+1 of 27 bushfire and 13 heatwave events respectively. Bushfire records
sourced from the Department of Environment, Land, Water and Planning
(DELWP). Heatwaves were computed using the 55\protect\textsuperscript{th}
percentile of values defined with \citet{Nairn2009} criteria. Climate
data obtained from the Tweentieth Century Reanalysis Project (20CR).
Areas marked with ``+'' show anomalies statistically significant
at the 5\% level. Cold fronts and persistent excess heat characterise
extreme fire weather and heatwaves weather respectively. \label{fig: Air temperature daily anomalies at 1000 hPa level during extreme bushfire and heatwave events in Victoria, Australia during the period 1961-2010}}


\end{figure}
 

The differences in relative humidity (1000 hPa level) are more revealing.
Southeastern Australia presents drier anomalies during heatwaves (see
Figure \ref{Relative humidity daily anomalies at 1000 hPa level during extreme bushfire and heatwave events in Victoria, Australia during the period 1961-2010}).
Above-average humidity inland is present around these dry anomalies
particularly in Queensland. During bushfires, the eastern coast appears
dry. The anomalies are elongated approximately following the Great
Dividing Range. Victoria remains drier after the cold front passage
for bushfire events. This indicates that such a dry cold front is
of relevance in triggering and spreading the fires. This can be seen
in figures \ref{fig: Air temperature daily anomalies at 1000 hPa level during extreme bushfire and heatwave events in Victoria, Australia during the period 1961-2010}c
and \ref{Relative humidity daily anomalies at 1000 hPa level during extreme bushfire and heatwave events in Victoria, Australia during the period 1961-2010}c. 

\begin{figure}[h]
\noindent \begin{centering}
\includegraphics[scale=0.65]{Chapter_4/Figures/RH_daily}
\par\end{centering}

\caption[Relative humidity daily anomalies at 1000 hPa level during extreme
bushfire and heatwave events in Victoria, Australia during the period
1961-2010]{Relative humidity daily anomalies at 1000 hPa level during extreme
bushfire and heatwave events in Victoria, Australia during the period
1961-2010. Panels a), b), c), d), e) and f) show anomalies for day
-1 to day +1 of 27 bushfire and 13 heatwave events respectively. Bushfire
records sourced from the Department of Environment, Land, Water and
Planning (DELWP). Heatwaves were computed using the 55\protect\textsuperscript{th}
percentile of values defined with \citet{Nairn2009} criteria. Climate
data obtained from the Tweentieth Century Reanalysis Project (20CR).
Areas marked with ``+'' show anomalies statistically significant
at the 5\% level. Southern Australia is dry during bushfires in Victoria
while the continent displays intense dry-wet contrasts linked to heatwaves
in this region. \label{Relative humidity daily anomalies at 1000 hPa level during extreme bushfire and heatwave events in Victoria, Australia during the period 1961-2010}}
\end{figure}


Geopotential height (500 hPa level) exhibits the most notable differences
(see Figure \ref{fig: Geopotential height daily anomalies at 1000 hPa level during extreme bushfire and heatwave events in Victoria, Australia during the period 1961-2010}).
This figure shows that heatwaves display a strong wave pattern that
leads to a blocking high over the Tasman Sea. This result agrees with
\citet{Pezza2012}. Bushfires occur in association with an intense
frontal activity signal. 

\begin{figure}[h]
\noindent \begin{centering}
\includegraphics[scale=0.65]{Chapter_4/Figures/HGT_daily}
\par\end{centering}

\caption[Geopotential height daily anomalies at 1000 hPa level during extreme
bushfire and heatwave events in Victoria, Australia during the period
1961-2010]{Geopotential height daily anomalies at 1000 hPa level during extreme
bushfire and heatwave events in Victoria, Australia during the period
1961-2010. Panels a), b), c), d), e) and f) show anomalies for day
-1 to day +1 of 27 bushfire and 13 heatwave events respectively..
Bushfire records sourced from the Department of Environment, Land,
Water and Planning (DELWP). Heatwaves were computed using the 55\protect\textsuperscript{th}
percentile of values defined with \citet{Nairn2009} criteria. Climate
data obtained from the Tweentieth Century Reanalysis Project (20CR).
Areas marked with ``+'' show anomalies statistically significant
at the 5\% level. Bushfires display intense frontal activity while
heatwaves show persistent anomalies. \label{fig: Geopotential height daily anomalies at 1000 hPa level during extreme bushfire and heatwave events in Victoria, Australia during the period 1961-2010}}


\end{figure}

Different weather patterns produce bushfire and heatwave events in
Victoria, Australia. The key feature of extreme fire weather in southeast
Australia are cold fronts \citep{Reeder1987,Reeder2015}. On the other
hand, a high-pressure system in the Tasman Sea is the key pattern
producing heatwaves in this region \citep{Pezza2012,Parker2014a}.
Their co-occurrence can produce catastrophic fire events \citep{Mills2005,Engel2013}.
However, after disentangling bushfire and heatwaves events, a synoptic
climatology shows that their differences prevail. 

Bushfires and heatwaves do not usually occur simultaneously in this
region. Extended periods of low relative humidity and the passage of cold fronts
contribute to bushfire occurrence. These patterns may have a greater
influence than short-term anomalous heat produced by heatwaves. These
conditions suggest that different dynamics drive these two types of
events. The discussion of the underlying mechanisms will focus on
the influence of modes of climate variability. 

Discussing geopotential height patterns can illustrate different dynamics
in action during bushfire and heatwaves. Overall, heatwaves show stronger
anomalies than bushfires. Bushfire events exhibit strong positive
anomalies between Australia and Antarctica. In contrast, heatwave
events present a precursor wave train in the Indian Ocean. 
The positive geopotential height anomalies over Antartica during fire weather days suggest that the intense frontal activity associated with extreme fire events is connected to a negative phase of the Southern Annular Mode. In contrast,  heatwave days show lower than normal pressure anomalies in the same region, which could be linked to the opposite phase of SAM. A negative SAM contributes to stable atmospheric conditions over southeastern Australia making heatwaves more intense. These differences suggest that the Southern Annular Mode and the strength of the westerlies over the Southern Ocean could play a major role driving these two types of events on a daily time-scale.

To expand on this discussion, it is important to analyse the sensitivity
of the results. Using an alternative bushfire database or heatwave
definition could yield different outcomes. Thus, Chapter 6 reproduces
the result shown in this chapter using alternative data.

\subsection{Seasonal large-scale circulation patterns}

On average, bushfire events are associated with warm sea surface temperature (SST)
anomalies over the tropical Pacific Ocean during December-January-February.
These anomalies are also present in the Indian Ocean (see Figure \ref{fig:Sea surface temperature anomalies of bushfire and heatwave seasons in December-January-February for the period 1961=0020132011}a).
In contrast, these warming anomalies are not present for heatwave
events (see Figure \ref{fig:Sea surface temperature anomalies of bushfire and heatwave seasons in December-January-February for the period 1961=0020132011}b).
The pattern suggests the influence of ``El Ni\~no'' events during
bushfires. Heatwaves occur on no clear ``El Ni\~no'' or ``La Ni\~na'' phase. 

\begin{figure}[h]
\noindent \begin{centering}
\includegraphics[scale=0.75]{Chapter_4/Figures/SSTs}
\par\end{centering}

\caption[Sea surface temperature (SST) anomalies composite of a) 18 bushfire
and b) 12 heatwave seasons occurred in Victoria, during the season December-January-February
(DJF) for the period 1961\textendash 2011]{Sea surface temperature (SST) anomalies composite of a) 18 bushfire
and b) 12 heatwave seasons occurred in Victoria, during the season December-January-February
(DJF) for the period 1961\textendash 2011. Bushfire records sourced
from the Department of Environment, Land, Water and Planning (DELWP).
Heatwaves were computed using the 55\protect\textsuperscript{th}
percentile of values defined with \citet{Nairn2009} criteria. Climate
data obtained from the Met Office Hadley Centre\textquoteright s sea
ice and sea surface temperature dataset (HadISST1) dataset. Shaded
areas show anomalies statistically significant at the 5\% level. Extreme
fire seasons in Victoria show a clear ``El Ni\~no'' pattern while
heatwaves display a neutral condition. \label{fig:Sea surface temperature anomalies of bushfire and heatwave seasons in December-January-February for the period 1961=0020132011}}


\end{figure}


A final approach in this investigation was to examine the seasonal
circulation. The focus was to the large-scale synoptic patterns. The
analysis comprised events occurred in spring (September-November)
and summer (December-February). 

Figure \ref{fig: Temperature anomalies (1000 hPa level) of bushfire and heatwave seasons in September-October-November for the period 1961=0020132011}
and figure \ref{fig:Temperature anomalies (1000 hPa level) of bushfire and heatwave seasons in December-January-February for the period 1961=0020132011}
show air temperature (1000 hPa level) anomalies composites for spring
and summer seasons respectively. Figure \ref{fig: Temperature anomalies (1000 hPa level) of bushfire and heatwave seasons in September-October-November for the period 1961=0020132011}a
shows an \textquoteleft El Ni\~no\textquoteright{} like pattern that
persists into the DJF season associated with bushfires (see Figure
\ref{fig:Temperature anomalies (1000 hPa level) of bushfire and heatwave seasons in December-January-February for the period 1961=0020132011}a).
In contrast, heatwave seasons suggest neutral ENSO conditions. This
pattern persists in the concurrent season (see Figure \ref{fig: Temperature anomalies (1000 hPa level) of bushfire and heatwave seasons in September-October-November for the period 1961=0020132011}b
and Figure \ref{fig:Temperature anomalies (1000 hPa level) of bushfire and heatwave seasons in December-January-February for the period 1961=0020132011}b). 

\begin{figure}[h]
\noindent \begin{centering}
\includegraphics[scale=0.75]{Chapter_4/Figures/T_SON}
\par\end{centering}

\caption[Seasonal composite for air temperature (T) anomalies (1000 hPa level)
for a) 18 bushfire and b) 12 heatwave seasons occurred in Victoria during the season
September-October-November (SON) for the period 1961\textendash 2011]{Seasonal composite for air temperature (T) anomalies (1000 hPa level)
for a) 18 bushfire and b) 12 heatwave seasons occurred in Victoria during the season
September-October-November (SON) for the period 1961\textendash 2011.
Bushfire records sourced from the Department of Environment, Land,
Water and Planning (DELWP). Heatwaves were computed using the 55\protect\textsuperscript{th}
percentile of values defined with \citet{Nairn2009} criteria. Climate
data obtained from the Tweentieth Century Reanalysis Project (20CR).
Shaded areas show anomalies statistically significant at the 5\% level.
Extreme fire seasons in Victoria show an ``El Ni\~no-like'' pattern
while heatwaves display a neutral condition. \label{fig: Temperature anomalies (1000 hPa level) of bushfire and heatwave seasons in September-October-November for the period 1961=0020132011}}


\end{figure}


\begin{figure}[h]
\noindent \begin{centering}
\includegraphics[scale=0.75]{Chapter_4/Figures/T_DJF}
\par\end{centering}

\caption[Seasonal composite for air temperature (T) anomalies (1000 hPa level)
for a) 18 bushfire and b) 12 heatwave seasons occurred in Victoria during the season
December-January-February (DJF) from 1961\textendash 2011]{Seasonal composite for air temperature (T) anomalies (1000 hPa level)
for a) 18 bushfire and b) 12 heatwave seasons occurred in Victoria during the season
December-January-February (DJF) from 1961\textendash 2011. Bushfire
records sourced from the Department of Environment, Land, Water and
Planning (DELWP). Heatwaves were computed using the 55\protect\textsuperscript{th}
percentile of values defined with \citet{Nairn2009} criteria. Climate
data obtained from the Tweentieth Century Reanalysis Project (20CR).
Shaded areas show anomalies statistically significant at the 5\% level.
Australia is a ``hot'' continent during extreme fire seasons in
Victoria while heatwaves display hot-cool contrasts. \label{fig:Temperature anomalies (1000 hPa level) of bushfire and heatwave seasons in December-January-February for the period 1961=0020132011}}
\end{figure}


Figure \ref{fig: Relative humidity anomalies (1000 hPa level) of bushfire and heatwave seasons in September-October-November for the period 1961=0020132011}
and figure \ref{fig: Relative humidity anomalies (1000 hPa level) of bushfire and heatwave seasons in December-January-February for the period 1961=0020132011}
display the relative humidity (1000 hPa level) anomalies. Figure \ref{fig: Relative humidity anomalies (1000 hPa level) of bushfire and heatwave seasons in September-October-November for the period 1961=0020132011}a
depicts a dry Australia. The negative anomalies are stronger in the
DJF season. Heatwave seasons display the opposite pattern. They present
positive anomalies in this region. Wet conditions become dominant
in a large area of the Australian continent. However, dry conditions
prevail in South Australia and Victoria during heatwaves (see Figure
\ref{fig: Relative humidity anomalies (1000 hPa level) of bushfire and heatwave seasons in September-October-November for the period 1961=0020132011}b
and Figure \ref{fig: Relative humidity anomalies (1000 hPa level) of bushfire and heatwave seasons in December-January-February for the period 1961=0020132011}b).
These results show that low relative humidity is a key seasonal precursor
of bushfires. It is also a concurrent element. The results suggest
relative humidity could be more important than temperature in explaining
bushfire occurrences. 

\begin{figure}[h]
\noindent \begin{centering}
\includegraphics[scale=0.75]{Chapter_4/Figures/RH_SON}
\par\end{centering}

\caption[Seasonal composite for relative humidity (RH) anomalies (1000 hPa
level) for a) 18 bushfire and b) 12 heatwave seasons occurred in Victoria during  the
season September-October-November (SON) from 1961\textendash 2011.]{Seasonal composite for relative humidity (RH) anomalies (1000 hPa
level) for a) 18 bushfire and b) 12 heatwave seasons occurred in Victoria during  the
season September-October-November (SON) from 1961\textendash 2011.
Bushfire records sourced from the Department of Environment, Land,
Water and Planning (DELWP). Heatwaves were computed using the 55\protect\textsuperscript{th}
percentile of values defined with \citet{Nairn2009} criteria. Climate
data obtained from the Tweentieth Century Reanalysis Project (20CR).
Shaded areas show anomalies statistically significant at the 5\% level.
Australia is dry during the antecedent fire season in Victoria while
wet conditions prevail in the continent linked to heatwaves in this
region.\label{fig: Relative humidity anomalies (1000 hPa level) of bushfire and heatwave seasons in September-October-November for the period 1961=0020132011}}


\end{figure}


\begin{figure}[h]
\noindent \begin{centering}
\includegraphics[scale=0.75]{Chapter_4/Figures/RH_djf}
\par\end{centering}

\caption[Seasonal composite for relative humidity (RH) anomalies (1000 hPa
level) for a) 18 bushfire and b) 12 heatwave seasons occurred in Victoria during the
season December-January-February from 1961\textendash 2011]{Seasonal composite for relative humidity (RH) anomalies (1000 hPa
level) for a) 18 bushfire and b) 12 heatwave seasons occurred in Victoria during the
season December-January-February from 1961\textendash 2011. Bushfire
records sourced from the Department of Environment, Land, Water and
Planning (DELWP). Heatwaves were computed using the 55\protect\textsuperscript{th}
percentile of values defined with \citet{Nairn2009} criteria. Climate
data obtained from the Tweentieth Century Reanalysis Project (20CR).
Shaded areas show anomalies statistically significant at the 5\% level.
During the concurrent fire season in Victoria Australia is dry. In
contrast, during heatwaves most of the continent is wet (except Southern
Australia). \label{fig: Relative humidity anomalies (1000 hPa level) of bushfire and heatwave seasons in December-January-February for the period 1961=0020132011}}


\end{figure}

On average, ENSO's influence over bushfire events in Victoria is moderate.
The SST anomalies in the tropical Pacific Ocean reveal warming conditions
(+ 0.5 \textsuperscript{o}C) during these events. These results are
consistent with the two databases used (see Chapter 6). Although,
only the NINO1+2 and the NINO3.4 regions showed statistical significance.
Paradoxically, the correlations between the bushfire time series and
the SOI were weak. This result suggests that the oceanic component
of ENSO has a greater influence over bushfires (see Chapter 5). With
a shorter period of analysis, \citet{Harris2013} also suggested that
ENSO's influence over fire weather is moderate. In contrast, ENSO
has a greater influence over bushfires in Tasmania \citep{Nicholls2007a} 
and the Northern Territory of Australia \citep{Harris2008}. 

On the other hand, heatwaves in Victoria do not seem linked to any
particular ENSO phase. Most heatwaves occur during neutral ENSO events,
a result that agrees with \citet{Boschat2014}. The results show this
is independent of the heatwave definition adopted (see Chapter 6). 

The role that the ocean might exert in the dynamics of bushfires and
heatwaves is of particular importance. The dynamics behind the differences
between these phenomena could be further investigated in this direction.
\citet{Sadler2012} proposed an experiment to measure the impact of
perturbed SSTs in the Tasman Sea in the lead-up to heatwaves. \citet{Fiddes2015}
discussed the role of SST in the enhancement of the cold front and
heatwave leading to ``Black Saturday''. It would be interesting
to attempt a similar study that investigates the influence of SST
in fire weather days not linked to heatwaves. 

On the continent, bushfires and heatwaves weather differ in the behaviour
of relative humidity. The preconditioning process leading to bushfires
is more intense compared to its heatwave counterpart. Victoria is
a dry region from September to February. A positive Indian Ocean Dipole
mode has been associated with this process \citep{Cai2009}. A slight
area of negative anomalies in Victoria is present in spring during
heatwaves. However, it is important to highlight how the rest of Australia
becomes more humid from spring to summer. 

The anomalous areas found in the composites could be used to design
a seasonal bushfire index. Chapter 5 proposes an index based on these
results. Such an index could contribute to anticipating extreme bushfire
seasons in Victoria. \citet{Harris2013} have already suggested this
possibility using climate variables and ENSO indices. Additionally,
the analysis of weather patterns conducive to lightning ignition could
demonstrate forecast skill. This gap is an exciting avenue for
further research. 

Finally, it is worth acknowledging that the relatively short period
of analysis is a limitation in this study. Heatwaves in this region
can be computed from the early years of the century. However, the
development of a robust bushfire database for the first half of the
20\textsuperscript{th} century in Victoria is a scientific gap. This
investigation requires proxy data such as fire history studies and
paleoclimate records. Unfortunately, this endeavour was out of the
scope of this research.

\section{Summary}

Bushfires and heatwaves are dangerous natural hazards in Australia.
Above-average air temperature anomalies characterise both of them.
However, they differ in several other climatological aspects. This
chapter compared the weather patterns associated with these events
in Victoria, Australia. The results showed that only 13\% of bushfires
co-occurred with heatwaves. Their main synoptic difference is 
the passage of a cold front and associated strong winds on the day that extreme bushfires start.  
Additionally, a humid Australia characterises
heatwaves while bushfires exhibit the opposite pattern. On average,
an ``El Ni\~no'' pattern dominates the tropical Pacific Ocean during
bushfires. On the other hand, heatwaves usually occur during neutral
ENSO conditions. These analyses used weather station as well as reanalysis
data. These findings suggest that using remote climate drivers patterns
linked to bushfires can increase the skill of fire weather forecasts
in Victoria.

The next chapter uses these patterns to develop a fire weather index.
The research also explored fire weather variability in Victoria in
the 20\textsuperscript{th} century. This long-term variability is
investigated using the FFDI as well as the proposed index. Finally,
Chapter 5 investigates climate-bushfire relationships. The aim is
to establish the skill to forecast extreme fire seasons in this region.