
\chapter*{Abstract}

This thesis investigated fire weather in Victoria, Australia and the
Ecuadorian Andes. The selection of these areas considered several
criteria. First of all, bushfires cause significant impacts in these
two regions. Victoria has endured some of the most catastrophic bushfire
events in Australian history (e.g. \textquotedblleft Black Friday\textquotedblright{}
(1939), \textquotedblleft Ash Wednesday\textquotedblright{} (1983),
\textquotedblleft Black Saturday\textquotedblright{} (2009)). On the
other hand, bushfires in Ecuador destroy every year large areas of
national parks in one of the most biodiverse countries in the world.
Secondly, the El Ni\~no-Southern Oscillation (ENSO) is a strong climate
driver in the two study areas. Finally, Victoria and Ecuador share
the \textit{Eucalyptus} as the dominant bushfire-prone species. 

The aim of this thesis is to better understand the drivers and evolution
of fire weather in these two regions of the Southern Hemisphere. Specifically,
it examined three aspects. First of all, it investigated fire weather
spatial patterns in Victoria and their relationship with associated
events like heatwaves. Subsequently, the study explored long-term
fire weather variability and changes. Finally, the investigation evaluated the influence
of ENSO and other climate drivers over fire weather.

The analyses used three groups of data: bushfire records, meteorological
and climate indices data. Consistent bushfire records were available
only for Victoria during the period 1961-2010. Additionally, the investigation
required observations from weather stations in Victoria and the Ecuadorian
Andes. This research also analysed reanalysis data from the Twentieth
Century Reanalysis Project (20CR) and the European Reanalysis of Global
Climate Observations ERA-Clim project (ERA-20C). The study had a stronger
emphasis on ENSO since it affects both regions. 

This research used two indices to represent fire
weather. The first index was the McArthur Forest Fire Danger Index
(FFDI). This Australian metric was designed for a \textit{Eucalyptus}
environment. Therefore, this investigation applied the FFDI for Victoria
and Ecuador. Additionally, this thesis proposes an alternative fire
weather index for Victoria: the ``Victorian Seasonal Bushfire Index''
(VSBI). The VSBI combines local meteorological variables and sea surface
temperature in ENSO regions to represent\textemdash and predict\textemdash extreme
fire weather. 

The investigation of fire weather in Victoria and the Ecuadorian Andes
yielded several findings. First of all, bushfire and heatwave weather
patterns display differences in Victoria. These comparisons used synoptic
climatologies with reanalysis data during the period 1961-2010. Additionally,
the investigation showed that Victoria experienced an increase in
fire danger during the period 1974-2010. There is also weaker evidence
suggesting an increasing trend since 1920. \textquotedblleft El
Ni\~no\textquotedblright{} events are the leading remote driver of fire
activity in Victoria. In fact, the incorporation of ENSO indicators
in a simple index (VSBI) shows skill to forecast extreme fire weather
in this region. For the Ecuadorian Andes, this research indicates
that its fire danger season (July-September) is longer than reported.
October and November also display \textquotedblleft high\textquotedblright{}
fire danger during the period 1997-2012. Finally, \textquotedblleft El
Ni\~no\textquotedblright{} events increase fire risk in the Ecuadorian
Andes. 
