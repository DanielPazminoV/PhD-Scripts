\appendix

\chapter{Drought factor calculation}

The drought factor formula, originally expressed as \textquotedblleft D\textquotedblright{}
by \citet{Griffiths1999} is:

\begin{equation}
D=10.5\left(1-\exp^{-\left(I+30/40\right)}\left(\frac{\left(41y^{2}+y\right)}{\left(40y^{2}+y+1\right)}\right)\right)
\end{equation}


Where:

$D$: Drought Factor. 

$I$: Soil dryness index (mm equivalent). 

The significant rain is any event that maximizes $y$, according to
the following cases shown in table \ref{tab:Significant-rain-event}

\begin{table}[h]
\caption[Significant rain event cases for drought factor calculation]{Significant rain event cases. Adapted from \citet{Griffiths1999}
\label{tab:Significant-rain-event}}


\noindent \centering{}%
\begin{tabular}{|c|c|c|}
\hline 
Case number & Significant rain factor & Conditions\tabularnewline
\hline 
\hline 
1 & $y=\frac{N^{1.3}}{\left(N^{1.3}+P-2\right)}$ & If N $\geq$1 and P > 2\tabularnewline
\hline 
2 & $y$= $\frac{0.8^{1.3}}{\left(0.8^{1.3}+P-2\right)}$ & If N=0 and P >2\tabularnewline
\hline 
3 & y = 1 & If P $\leq$ 2\tabularnewline
\hline 
\end{tabular}
\end{table}


Where: 

$N$: Time since the rain event in days. 

$P$: Rainfall in mm during an event. 


\chapter{Bushfire and heatwave dates}

\newpage{}


\section{Bushfire and heatwave dates included in the analysis}


\subsection{Bushfire dates }


\subsubsection{Department of Environment, Land, Water and Planning of Victoria,
Australia }

\begin{table}[H]
\caption{Bushfire records from the Department of Environment, Land, Water and
Planning used to analyse fire weather spatial patterns in Victoria,
Australia. Bushfire events that occurred simultaneously with heatwaves
defined using Nairn et al. (2009) definition were excluded. The table
shows the start date 27 bushfire events. \label{tab:Bushfire records from the Department of Environment, Land, Water and Planning used to analyse fire weather spatial patterns in Victoria, Australia}}


\centering{}%
\begin{tabular}{|c|c|c|c|c|}
\hline 
\multicolumn{5}{|c|}{\textbf{Bushfire event dates (included in the analysis)}}\tabularnewline
\hline 
\multicolumn{5}{|c|}{Department of Environment, Land, Water and Planning}\tabularnewline
\hline 
\hline 
17-Jan-1965 & 12-Feb-1977 & 31-Jan-1983 & 3-Jan-2003 & 11-Jan-2007\tabularnewline
\hline 
8-Jan-1969 & 7-Jan-1979 & 14-Jan-1985 & 8-Jan-2003 & 17-Dec-2009\tabularnewline
\hline 
14-Dec-1972 & 28-Dec-1980 & 25-Feb-1995 & 11-Jan-2005 & 1-Feb-2011\tabularnewline
\hline 
9-Jan-1975 & 15-Jan-1981 & 1-Jan-1998 & 1-Dec-2006 & \tabularnewline
\hline 
8-Feb-1975 & 13-Dec-1982 & 28-Jan-1999 & 10-Dec-2006 & \tabularnewline
\hline 
4-Feb-1977 & 16-Feb-1983 & 17-Dec-2002 & 20-Jan-2006 & \tabularnewline
\hline 
\end{tabular}
\end{table}



\subsubsection{Risk Frontiers}

\begin{table}[H]
\caption{Bushfire records from the Risk Frontiers research centre's database.
Bushfire events that occurred simultaneously with heatwaves defined
using Pezza et al. (2012) definition were excluded. The table shows
the start date 41 bushfire events.}


\centering{}%
\begin{tabular}{|c|c|c|c|c|}
\hline 
\multicolumn{5}{|c|}{\textbf{Bushfire event dates}}\tabularnewline
\hline 
\hline 
\multicolumn{5}{|c|}{Risk Frontiers}\tabularnewline
\hline 
29-Jan-1961 & 1-Jan-1973 & 1-Feb-1983 & 2-Dec-1998 & 16-Jan-2007\tabularnewline
\hline 
17-Jan-1965 & 2-Jan-1973 & 7-Feb-1983 & 26-Jan-2003 & 30-Jan-2009\tabularnewline
\hline 
8-Feb-1967 & 3-Jan-1973 & 16-Feb-1983 & 11-Jan-2005 & 7-Feb-2009\tabularnewline
\hline 
31-Jan-1968 & 24-Jan-1973 & 14-Jan-1985 & 31-Dec-2005 & 23-Feb-2009\tabularnewline
\hline 
5-Feb-1968 & 12-Feb-1977 & 27-Dec-1990 & 19-Jan-2006 & 10-Jan-2010\tabularnewline
\hline 
18-Feb-1968 & 15-Jan-1978 & 25-Feb-1995 & 22-Jan-2006 & \tabularnewline
\hline 
19-Feb-1968 & 22-Dec-1980 & 21-Jan-1997 & 10-Dec-2006 & \tabularnewline
\hline 
8-Jan-1969 & 1-Jan-1981 & 1-Jan-1998 & 22-Dec-2006 & \tabularnewline
\hline 
8-Jan-1971 & 15-Feb-1982 & 18-Feb-1998 & 12-Jan-2007 & \tabularnewline
\hline 
\end{tabular}
\end{table}



\subsection{Heatwave dates }


\subsubsection{Nairn et al. (2009) }

\begin{table}[H]
\caption{Heatwave dates computed using Nairn et al. (2009) definition. Heatwave
events that occurred simultaneously with fires recorded in the Department
of Environment, Land, Water and Planning database were excluded. The
table shows the start date 13 heatwave events. \label{tab:Heatwave dates computed using Nairn et al. (2009) definition}}


\centering{}%
\begin{tabular}{|c|c|c|c|}
\hline 
\multicolumn{4}{|c|}{\textbf{Heatwave events }}\tabularnewline
\hline 
\hline 
\multicolumn{4}{|c|}{Nairn et \textit{al.} (2009) definition}\tabularnewline
\hline 
14-Jan-1962 & 29-Jan-1993 & 18-Feb-2001 & 29-Jan-2011\tabularnewline
\hline 
18-Jan-1982 & 11-Feb-1995 & 14-Feb-2007 & \tabularnewline
\hline 
6-Feb-1982 & 18-Feb-1997 & 8-Jan-2008 & \tabularnewline
\hline 
6-Jan-1988 & 2-Feb-2000 & 7-Feb-2010 & \tabularnewline
\hline 
\end{tabular}
\end{table}



\subsubsection{Pezza et al. (2012) }

\begin{table}[H]
\caption{Heatwave dates computed using Pezza et al. (2012) definition. Heatwave
events that occurred simultaneously with fires recorded in the Risk
Frontier's database were excluded. The table shows the start date
13 heatwave events. \label{tab:Heatwave dates computed using Pezza et al. (2012) definition}}


\centering{}%
\begin{tabular}{|c|c|c|c|}
\hline 
\multicolumn{4}{|c|}{\textbf{Heatwave events }}\tabularnewline
\hline 
\hline 
\multicolumn{4}{|c|}{Pezza et \textit{al.} (2009) definition}\tabularnewline
\hline 
16-Feb-1968 & 15-Dec-1988 & 2-Feb-2000 & 30-Jan-2011\tabularnewline
\hline 
18-Feb-1969 & 31-Jan-1993 & 8-Feb-2000 & \tabularnewline
\hline 
19-Dec-1972 & 12-Jan-1998 & 28-Dec-2002 & \tabularnewline
\hline 
18-Jan-1973 & 23-Dec-1998 & 28-Jan-2009 & \tabularnewline
\hline 
\end{tabular}
\end{table}



\section{Bushfire and heatwave dates excluded from the analysis}


\subsection{Bushfire dates}


\subsubsection{Department of Environment, Land, Water and Planning of Victoria,
Australia }

\begin{table}[H]
\caption{Bushfire records from the Department of Environment, Land, Water and
Planning excluded from the analysis because they occurred simultaneously
with heatwave events defined using Nairn et al. (2009) criteria. }


\centering{}%
\begin{tabular}{|c|}
\hline 
\textbf{Bushfire event dates (excluded from the analysis)}\tabularnewline
\hline 
\hline 
Department of Environment, Land, Water and Planning\tabularnewline
\hline 
4-Feb-2009\tabularnewline
\hline 
6-Feb-2009\tabularnewline
\hline 
7-Feb-2009\tabularnewline
\hline 
8-Feb-2009\tabularnewline
\hline 
\end{tabular}
\end{table}



\subsubsection{Risk Frontiers }

\begin{table}[H]
\caption{Bushfire records from Risk Frontiers research centre's database excluded
from the analysis because they occurred simultaneously with heatwave
events defined using Pezza et al. (2012) criteria. }


\centering{}%
\begin{tabular}{|c|}
\hline 
\textbf{Bushfire event dates (excluded from the analysis)}\tabularnewline
\hline 
\hline 
Risk Frontiers\tabularnewline
\hline 
22-Feb-1967\tabularnewline
\hline 
1-Feb-2011\tabularnewline
\hline 
\end{tabular}
\end{table}



\subsection{Heatwave dates}


\subsubsection{Nairn et al. (2009) }

\begin{table}[H]
\caption{Heatwave dates computed using Nairn et \textit{al.} (2009) definition
that were excluded from the analysis because they occurred simultaneously
with fires recorded in the Department of Environment, Land, Water
and Planning database. }


\centering{}%
\begin{tabular}{|c|}
\hline 
\textbf{Heatwave event dates (excluded from the analysis)}\tabularnewline
\hline 
\hline 
Nairn et al. (2009)\tabularnewline
\hline 
3-Feb-2009\tabularnewline
\hline 
\end{tabular}
\end{table}



\subsubsection{Pezza et al. (2012)}

\begin{table}[H]
\caption{Heatwave dates computed using Pezza et \textit{al.} (2012) definition
that were excluded from the analysis because they occurred simultaneously
with fires recorded in the Risk Frontiers database. }


\centering{}%
\begin{tabular}{|c|}
\hline 
\textbf{Heatwave event dates (excluded from the analysis)}\tabularnewline
\hline 
\hline 
Pezza et al. (2012)\tabularnewline
\hline 
21-Feb-1967\tabularnewline
\hline 
13-Feb-1982\tabularnewline
\hline 
20-Jan-2006\tabularnewline
\hline 
\end{tabular}
\end{table}



\chapter{Weather stations details}

\newpage{}


\section{Victoria}

\begin{table}[h]
\caption{Victorian weather stations selected to compute hetwave dates. Data
sourced from the Australian Climate Observations and Reference Network-Surface
Air Temperature (ACORN-SAT) of the Australian Bureau of Meteorology.
\label{tab:Victorian weather stations selected to compute hetwave dates} }


\begin{centering}
\begin{tabular}{|c|c|c|c|c|}
\hline 
\textbf{Site name} & \textbf{Site number} & \textbf{Latitude} & \textbf{Longitude} & \textbf{Elevation (m)}\tabularnewline
\hline 
\hline 
Melbourne Regional Office & 086071 & 37.81\textsuperscript{o}S & 144.97\textsuperscript{o}E & 31\tabularnewline
\hline 
Mildura & 076031 & 34.24\textsuperscript{o}S & 142.09\textsuperscript{o}E & 50\tabularnewline
\hline 
Nhill & 078015 & 36.31\textsuperscript{o}S & 141.65\textsuperscript{o}E & 139\tabularnewline
\hline 
Sale & 085072 & 38.12\textsuperscript{o}S & 147.13\textsuperscript{o}E & 5\tabularnewline
\hline 
\end{tabular}
\par\end{centering}

\end{table}


\begin{table}[h]


\caption[Victorian weather stations selected to investigate fire weather]{Victorian weather stations selected to investigate fire weather. Data
sourced from the Australian Bureau of Meteorology. \label{tab:Victorian weather stations selected to investigate fire weather} }


\centering{}%
\begin{tabular}{|c|c|c|c|c|}
\hline 
\textbf{Site name} & \textbf{Site number} & \textbf{Latitude} & \textbf{Longitude} & \textbf{Elevation (m)}\tabularnewline
\hline 
\hline 
Bendigo & 081003 & 36.76\textsuperscript{o}S & 144.28\textsuperscript{o}E & 208\tabularnewline
\hline 
Laverton & 087031 & 37.86\textsuperscript{o}S & 144.76\textsuperscript{o}E & 20\tabularnewline
\hline 
Melbourne Airport & 086282 & 37.68\textsuperscript{o}S & 144.84\textsuperscript{o}E & 113\tabularnewline
\hline 
Mildura & 076031 & 34.24\textsuperscript{o}S & 142.09\textsuperscript{o}E & 50\tabularnewline
\hline 
Nhill & 078015 & 36.31\textsuperscript{o}S & 141.65\textsuperscript{o}E & 139\tabularnewline
\hline 
Omeo & 083025 & 37.10\textsuperscript{o}S & 147.60\textsuperscript{o}E & 689\tabularnewline
\hline 
\end{tabular}
\end{table}



\section{Ecuador}

\begin{table}[h]
\noindent \begin{centering}
\caption[Ecuadorian weather stations selected to investigate fire
weather]{Ecuadorian weather stations selected to investigate fire
weather. Data sourced from the Ecuadorian Institute for Meteorology
and Hydrology (INAMHI). \label{tab:Table-1:Ecuadorian weather stations selected to investigate fire weather}}

\par\end{centering}

\noindent \centering{}%
\begin{tabular}{|c|c|c|c|c|}
\hline 
\textbf{Site name} & \textbf{Site number} & \textbf{Latitude} & \textbf{Longitude} & \textbf{Elevation (m)}\tabularnewline
\hline 
\hline 
La Tola & M002 & 0.23\textsuperscript{o}S & 78.37\textsuperscript{o}E & 2480\tabularnewline
\hline 
Rumipamba-Salcedo & M004 & 1.02\textsuperscript{o}S & 78.59\textsuperscript{o}E & 2685\tabularnewline
\hline 
Ca\~nar & M031 & 2.55\textsuperscript{o}S & 78.95\textsuperscript{o}E & 3083\tabularnewline
\hline 
La Argelia-Loja & M033 & 4.03\textsuperscript{o}S & 79.20\textsuperscript{o}E & 2160\tabularnewline
\hline 
Palmas-Azuay & M045 & 2.72\textsuperscript{o}S & 78.63\textsuperscript{o}E & 2400\tabularnewline
\hline 
San Gabriel & M103 & 0.60\textsuperscript{o}N & 77.82\textsuperscript{o}E & 2860\tabularnewline
\hline 
\end{tabular}
\end{table}



\chapter{Climate-bushfire relationships including the Antartic Oscillation
Index }

\newpage{}


\section{Spring climate-bushfire relationships}

\begin{figure}[H]
\begin{centering}
\includegraphics[scale=0.85]{Appendix/Figures/SON-SON_1979_2009}
\par\end{centering}

\caption[Pearson correlation coeficients for climate-bushfire relationships
for September-October-November (SON) in Victoria during the period
1979-2009]{Pearson correlation coeficients for climate-bushfire relationships
for September-October-November (SON) in Victoria during the period
1979-2009. Data source: Hadley Centre Sea Ice and Sea Surface Temperature
data set (HadISST) and weather stations in Victoria (WS). \label{fig:Pearson correlation coeficients for climate-bushfire relationships for September-October-November (SON) in Victoria during the period 1979-2009}}


\end{figure}



\section{Summer climate-bushfire relationships}

\begin{figure}[H]
\begin{centering}
\includegraphics[scale=0.75]{Appendix/Figures/DJF-DJF_1980_2010}
\par\end{centering}

\caption[Pearson correlation coeficients for climate-bushfire relationships
for December-January-February (DJF) in Victoria during the period
1980-2010]{Pearson correlation coeficients for climate-bushfire relationships
for December-January-February (DJF) in Victoria during the period
1980-2010. Data source: Hadley Centre Sea Ice and Sea Surface Temperature
data set (HadISST) and weather stations in Victoria. \label{fig:Pearson correlation coeficients for climate-bushfire relationships for December-January-February (DJF) in Victoria during the period 1980-2010}}


\end{figure}



\section{Spring to summer climate-bushfire relationships}

\begin{figure}[H]
\begin{centering}
\includegraphics[scale=0.65]{Appendix/Figures/SON-DJF_1979_2010}
\par\end{centering}

\caption[Pearson correlation coeficients for climate-bushfire relationships
from September-October-November (SON) to December-January-February
(DJF) in Victoria, Australia during the period 1979-2010]{Pearson correlation coeficients for climate-bushfire relationships
from September-October-November (SON) to December-January-February
(DJF) in Victoria, Australia during the period 1979-2010. Panel a)
shows the results for climate variables, fire weather and bushfire
activity. Panel b) shows the results for climate indices, fire weather
and bushfire activity Data source: Hadley Centre Sea Ice and Sea Surface
Temperature data set (HadISST) and weather stations in Victoria (WS).
\label{fig:Pearson correlation coeficients for climate-bushfire relationships from September-October-November (SON) to December-January-February (DJF) in Victoria, Australia during the period 1979-2010}}


\end{figure}

