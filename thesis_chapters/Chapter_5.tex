
\chapter{Fire weather variability and seasonal forecast skill in Victoria, Australia}
\newpage{}


\section{Introduction}

Chapter 4 displayed seasonal composites of fire weather variables.
These composites link climate to extreme bushfire occurrences in Victoria.
This analysis reflects average\textemdash large-scale\textemdash climate
patterns associated with bushfire activity. The composites showed
anomalies with consistent features from spring\textemdash September-October-November\textemdash to
summer\textemdash December-January-February\textemdash . Overall,
the anomalies are moderate to strong and significant in Victoria and the tropical
Pacific Ocean. 

The aim of this chapter is to complement the investigation of average
spatial patterns. Chapter 5 investigates the long-term variability
of fire weather in Victoria. Fire weather was investigated using seasonal
metrics of the McArthur Forest Fire Danger Index. This index quantifies
bushfire danger using meteorological data. Additionally, this chapter
proposes an alternative bushfire index: the \textquotedblleft Victorian
Seasonal Bushfire Index\textquotedblright{} (VSBI). The design of
this index used the anomalous regions found in Chapter 4.

Finally, this chapter explores the capability of the VSBI to forecast
fire weather\textemdash and fire activity\textemdash . The investigation
required four groups of data: bushfire activity records, climate variables,
fire weather and climate indices data. Each group comprises several
variables that were linearly correlated for spring, summer and from
spring to summer. Using weather station data, the VSBI shows a strong
forecasting skill and consistent climate-bushfire relationships. On
the other hand, results using the Twentieth Century Reanalysis data
for the Victorian region show the limitations of this dataset to capture
these relationships. 


\section{Data and methodology}


\subsection{Data}


\subsubsection{Bushfire activity records}

This experiment analysed the same bushfire activity datasets used
in the previous chapter. Additionally, the definitions of bushfire
event remain the same. A bushfire event in Victoria's Department of
Environment, Land, Water and Planning (DELWP) dataset refers to any
event that burnt more than 10,000 hectares. Risk Frontiers (RF) database
defines a bushfire as the consolidation of several bushfire\textemdash newspaper\textemdash reports
in Victoria. Using these definitions, both datasets are comparable
as demonstrated in Chapter 4. The period of validated bushfire records
data is 1961-2010. December-January-February (DJF) is the selected
bushfire season in this investigation, since extreme bushfire occurrences
are scarce in other months. For example, the DELWP database registered
31 events during the DJF season over the period 1961-2011. The other
three seasons of the year registered only 11 events during the same
period. The investigation before this period requires alternative
variables such as fire weather metrics. 


\subsubsection{Fire weather metrics}

The quantification of fire weather required the computation of the
McArthur Forest Fire Danger Index (FFDI) and the ``Victorian Seasonal
Bushfire Index'' (VSBI) in Victoria (Section \ref{sub:Formulation-VSBI}
describes the VSBI formulation). The FFDI provides daily values of
bushfire danger based on meterological data. However, this research
used a seasonal perspective approach. Thus, it was necessary to compute
seasonal metrics of fire weather based on the FFDI. There are two
common seasonal metrics of this index. The first metric is a cumulative
sum of daily FFDI values (FFDI\textsubscript{cum}). The second metric
is the number of days with FFDI values greater than an arbitrary threshold.
For this investigation the chosen threshold was 25 (FFDI\textsubscript{>25}).
Values above this threshold reflect days with very high to catastrophic
bushfire danger. The data used for the analyses were seasonal values
during spring (September-October-November) and summer (December-January-February-March)
of FFDI\textsubscript{cum} and FFDI\textsubscript{>25} in Victoria.
The FFDI computation comprised the period 1920-2010. This calculation
required finding the date with the highest precipitation to set the
soil moisture deficit to zero. This factor is part of the Keet-Byram
Drought Index (KBDI) calculation \citep{KeetchJJ} (see Chapter 2). 

On the other hand, the VSBI requires normalized climate data averaged
for each variable (summed for precipitation). A time series of VSBI
values was a third input of fire weather in the analyses. Section
\ref{sub:Formulation-VSBI} describes the VSBI formulation in detail.
The analyses used climate data to compute seasonal fire weather indices.


\subsubsection{Climate variables}

The computation of fire weather indices required daily values of climate
variables. The calculations with weather station (WS) data used 3
pm records. This is the approximate time when the most severe fire
weather conditions occur \citep{Lucas2010}. Six Victorian weather
stations were selected for this analysis (see Figure \ref{fig:Location of meteorological stations used to compute fire weather in Victoria}
and Appendix B). These stations provide the best available information
for the computation of the indices. Seasonal fire weather in Victoria
was also estimated using reanalysis data. Using the 20CR and HadISST
data allowed to extend the calculations to the first half of the 20\textsuperscript{th}
century. The 20CR data selected was the ensemble mean at the 1000
hPa level at 4 pm (6:00 GMT) . Climate data was also used to compute
climate indices.

\begin{figure}[h]
\noindent \begin{centering}
\includegraphics[scale=0.50]{Chapter_5/Figures/Maps/Location_ws.png}
\par\end{centering}

\caption[Location of meteorological stations used to compute fire weather in
Victoria]{Location of meteorological stations used to compute fire weather in
Victoria \label{fig:Location of meteorological stations used to compute fire weather in Victoria}}


\end{figure}



\subsubsection{Climate indices}

This investigation also explored the influence of several climate
modes of variability over bushfires in Victoria. The El Ni\~no Southern
Oscillation (ENSO), Indian Ocean Dipole (IOD) and Southern Annular
Mode (SAM) are among the most important for this region \citep{Gong1999,Williams2001,Cai2009,Risbey2009b,Ho2011,Ummenhofer2011,White2013}.
Seasonal weather composites suggested a strong influence of ENSO over
bushfire occurrences in Victoria. Thus, the analysis has a higher
emphasis over this driver. 

The analysis used three ENSO indices, and one index for the IOD and
SAM respectively. The Southern Oscillation Index (SOI), NINO3.4 and
NINO1+2. The Dipole Mode Index (DMI) and Antartic Oscillation (AAO)
revealed the relationships with IOD and SAM respectively. The analysis
used climate indices data from several institutions: Australian Bureau
of Meteorology (SOI), National Oceanic and Atmospheric Administration
(NINO3.4, AAO), Japan Agency for Marine-Earth Science and Technology
(DMI). The climatologies used in the calculation of these indices
were were 1951-2000 (NINO3.4.), 1958-2010 (DMI), and 1979-2000 (AAO).
The NINO1+2 index was computed using the HadISST data. The climatology
for this index was 1911-2011 


\subsection{Methodology}

The investigation of fire weather variability and forecasting skill
in Victoria required four analyses: validation of bushfire activity
data, computation of the FFDI, formulation and calculation of the
VSBI, and calculation of linear correlations. 


\subsubsection{Validation of bushfire activity data}

A comparison of the two bushfire datasets available\textemdash DELWP
and RF\textemdash was necessary to confirm their consistency. Chapter
4 describes the details of this process.


\subsubsection{Computation of the McArthur Forest Fire Danger Index}

The investigation used the FFDI to explore seasonal fire weather in
Victoria. This index is commoly used in Australia to assess fire behaviour
\citep{Lucas2010}. The computation of the FFDI requires meteorological
data reported on individual weather stations. The regional average
of each FFDI variable was used to compute values for Victoria. On
the other hand, averaging the gridded reanalysis data over Victoria
yielded regional values for the computation of the index. The 20CR
and WS data comprise daily values. However, this investigation aimed
to analyse fire weather with a seasonal perspective. 

The analysis used two seasonal metrics of FFDI. The first metric was
the seasonal sum of daily FFDI values (FFDI\textsubscript{cum}).
The second metric was the number of days in a season that an FFDI
exceeds a value of 25 (FFDI\textsubscript{>25}). Values above this threshold represent
days with a fire danger rating ranging from "very high" to "catastrophic" ("code red").
In both cases, the seasons considered were September-October-November (SON) and December-January-February
(DJF). Additionally, a comparison of reanalysis and weather station
data revealed the need for bias-correction. Probability density functions
(PDFs) and seasonal time series allowed to make these comparisons
between the two types of data. The discussion section \ref{sub:Seasonal-fire-weather-variability}
further examines the biases found in the 20CR data. 

The approach used to bias-correct the reanalysis dataset was to apply
linear scaling techniques. These methods require the calculation of a scaling factor. 
For each variable, the scaling factor allowed centring the mean of the model dataset that showed biases to the mean of the weather station data distribution. 
The scaling factors were calculated considering the deviations in the PDFs and time series plots that compared the two types of data. 
A visual appraisal of these plots guided the decision to test the best linear scaling method.    
  
Maximum temperature and wind speed required additive and multiplicative linear scaling respectively. 
The additive linear scaling comprised averaging the 20CR and WS datasets and subtracting their means to obtain the scaling factor. 
The multiplicative linear scaling required dividing the means instead of subtracting them. 
In the case of maximum temperature, the scaling factor was 2.36. This value was added to each value in the
20CR reanalysis dataset because the 20CR data underestimates maximum temperature in comparison to WS observations. 
For wind speed, the scaling factor was 1.37. Since the time series plots shows that reanalysis data overestimates wind speed, each value of this dataset was
divided by the scaling factor.               
 
On the other hand, relative humidity is a calculated variable. Therefore, the best
approach was to bias-correct the variables required for its computation.
The required variables are maximum temperature and specific humidity.
Specific humidity required additive linear scaling. The equations
used to calculate relative humidity are: 

\begin{equation}
q\equiv\frac{m_{v}}{m_{v}+m_{d}}=\frac{w}{w+1}\approx w
\end{equation}


\begin{equation}
ws\equiv\frac{m_{vs}}{m_{d}}=\frac{e_{s}R_{d}}{R_{v}\left(p-e_{s}\right)}\thickapprox0.622\frac{e_{s}}{p}
\end{equation}


\begin{equation}
e_{s}\left(T\right)=e_{s()}exp\left[\left(\frac{L_{v}\left(T\right)}{R_{v}}\right)\left(\frac{1}{T_{o}}-\frac{1}{T}\right)\right]\thickapprox611exp\left(\frac{17.67\left(T-T_{o}\right)}{T-29.65}\right)
\end{equation}


\begin{equation}
RH=100\frac{w}{w_{s}}\thickapprox0.263pq\left[exp\left(\frac{17.67\left(T-To\right)}{T-29.65}\right)\right]^{-1}
\end{equation}


Where:

$q:$ specific humidity or the mass mixing ratio of water vapor to
total air (dimensionless)

$m_{v}:$ specific mass of water vapor (Kg)

$m_{vs}:$specific mass of water vapor at equilibrium (Kg)

$m_{d}$: specific mass of dry air (Kg)

$w:$mass mixing ratio of water vapor to dry air at equilibrium (dimensionless)

$w_{s}:$mass mixing ratio of water vapor to dry air at equilibrium
(dimensionless)

$e_{s}(T):$saturation vapor pressure (Pa)

$e_{s()}:$saturation vapor pressure at $T_{o}$ (Pa)

$R_{d}:$specific gas constant for dry air (J $Kg^{-1}$ $K^{-1}$)

$R_{v}:$specific gas constant for water vapor (J $Kg^{-1}$ $K^{-1}$)

$p:$pressure (Pa)

$L_{v}(T)$: specific enthalpy of vaporization J $Kg^{-1}$

$T$: temperature (K)

$T_{o}$: reference temperature (273.16 K)

Precipitation was not bias-corrected. The bias-correction attempts
for this variable were not successful. However, the FFDI is less sensitive
to precipitation changes \citep{Dowdy2009}. 


\subsubsection{Formulation of the Victorian Seasonal Bushfire Index \label{sub:Formulation-VSBI}}

The composites presented in Chapter 4 revealed the seasonal patterns
of two fire weather variables. Temperature and relative humidity displayed
significant anomalies over Victoria (33S-36S; 145E-150E). Additionally,
the NINO1+2 (0-10S, 90W-80W) and NINO3.4 (5N-5S; 170W-120W) regions
showed strong and also statistical significant anomalies (see Figure
\ref{fig: Victorian Seasonal Bushfire Index (VSBI) regions}). These
anomalies are a manifestation of regional and global climate dynamics
linked to bushfire occurrences in Victoria. Therefore, the analysis
tested these regions for the VSBI formulation. Intial tests yielded
similar results using any of the two ``NINO'' regions. This chapter
will show the results using NINO1+2 (SST1.2) region. Additionally,
the VSBI includes precipitation since it is an important seasonal
fire weather variable.

\begin{figure}[h]
\noindent \begin{centering}
\includegraphics[scale=0.75]{Chapter_5/Figures/Maps/VSBI_location_areas}
\par\end{centering}

\caption[Victorian Seasonal Bushfire Index (VSBI) regions]{Victorian Seasonal Bushfire Index (VSBI) regions. The plot shows these
areas over the seasonal composites of surface temperature (T) and
sea surface temperature (SST) anomalies during bushfires in Victoria.
The composites display anomalies for the season December-January-February
during the period 1961-2010. \label{fig: Victorian Seasonal Bushfire Index (VSBI) regions}}


\end{figure}


The VSBI represents a simplified relationship between these variables.
This relationship links large-scale and local climate processes that
produce severe bushfire conditions. The VSBI assumes a linear relationship
between the variables. Temperature, relative humidity and precipitation
have the effect of preconditioning bushfire occurrences drying fuel
loads \citep{BoM2009}. This was the main criteria for their selection.
The index also incorporates SST in the NINO1+2 region as another variable.
The seasonal patterns showed in Chapter 4 revealed that this variable
is an important predictor of bushfire activity in Victoria. The formulation
of the VSBI comprises the aggregation of sub-indices based on these
variables. 

The first step in computing the VSBI was to use the anomalies from
the selected regions to calculate spatial averages. Subsequently,
these daily means were averaged for the seasons SON and DJF respectively.
The climatologies used were 1900-2010 for the 20CR and 1911-2011 for
HadISST data. The computation comprised the period 1920-2010. On the
other hand, the daily values of the six weather stations were averaged
to calculate regional anomalies. This procedure allowed the calculation of
the VSBI sub-indices using weather station data. The available period
of WS data was 1974-2010. Spatial average of the bias-corrected
20CR data allowed to replicate the analysis using this model. The
time series generated were divided by their standard deviation to
normalize them. Finally, the VSBI is represented in the following
in Equation \ref{eq:VSBI}. Figure \ref{fig:VSBI calculation procedure}
illustrates the VSBI calculation procedure.

\noindent \begin{center}
\begin{equation}
VSBI=\frac{T}{\sigma_{T}}-\frac{RH}{\sigma_{RH}}-\frac{P}{\sigma_{P}}+\frac{SST_{1+2}}{\sigma_{SST_{1+2}}}\label{eq:VSBI}
\end{equation}

\par\end{center}

\begin{figure}[h]
\caption{VSBI calculation procedure \label{fig:VSBI calculation procedure} }


\noindent \centering{}\includegraphics[scale=0.4]{Chapter_5/Figures/VSBI/VSBI_computation_Diagram}
\end{figure}



\subsubsection{Calculation of linear correlations}

The investigation used linear correlations to analyse the relationship
between bushfire activity records, seasonal fire weather metrics,
meteorological variables and modes of climate variability indices.
The correlations were computed for spring (SON-SON), summer (DJF-DJF),
and from spring to summer (SON-DJF). The analysis used bias-corrected
reanalysis data for the periods 1974-2010, 1961-2010, and 1920-2010.
These periods comprise the years with available weather station, bushfire
activity and FFDI data\textemdash computed with reanalysis data\textemdash respectively.
The figures highlight the statistically significant results at the
5\% level. The results using the Antartic Annular Oscillation Index
(AAO) are included in Appendix C. The Southern Annular Mode\textemdash measured
by the AAO\textemdash showed weak correlations with most variables
in a preliminary analysis undertaken for the period 1980-2010. 


\section{Results}

This section depicts the long-term fire weather variability in Victoria
represented by the FFDI and the VSBI. Additionally, it describes the
basic relationships between the data used in this investigation during
the antecedent (SON) and simultaneous (DJF) bushfire seasons. This
section also shows the fire weather\textemdash and fire activity\textemdash seasonal
forecasting skill from SON to DJF 


\subsection{Seasonal fire weather variability}


\subsubsection{McArthur Forest Fire Danger computation}

Figure \ref{fig:Probability density functions of orginal and bias-corrected Forest Fire Danger Index results for Victoria, Australia during the period 1974-2010}
shows FFDI probability density functions calculated with weather station
and reanalysis data (original and bias-corrected). The plot shows
that the FFDI daily mean was accurately adjusted. Figure \ref{fig: Cumulative Forest Fire Danger Index time series during December-January-February in Victoria, Australia for the period 1974-2010}
shows two times series of seasonal FFDI in Victoria. Panel a) shows
the results using FFDI\textsubscript{cum}with weather stations and
original reanalysis data. Panel b) displays the FFDI\textsubscript{cum}
comparison with bias-corrected 20CR data. These calculations are for
the season DJF during the period 1974-2010. Before this period weather
stations do not have enough high quality wind data to compute the FFDI. Weather station
data was necessary to bias-correct the FFDI results computed with
reanalysis data. 

Fire weather computed using WS data shows Victoria with more extreme
seasons compared to the original (without bias-correction) results
using reanalysis data. In spite of this, the results show a similar
variability for the two seasonal FFDI metrics. In panel a), the FFDI\textsubscript{cum}
time series show a moderate linear correlation (r=0.47, p=0.00, n=37).
However, on an inter-annual basis they differ on magnitude. Therefore,
a bias-correction of reanalysis data was required. Panel b) shows
the results with bias-correction (r=0.45, p=0.06, n=37). On the other
hand, Figure \ref{fig: Number of days with Forest Fire Danger Index greater than 25 time series during December-January-February in Victoria, Australia for the period 1974-2010}
shows the same comparison using time series of FFDI\textsubscript{>25}.
The correlation coefficients for these time series are a) r=0.33,
p=0.05, n=37 and b) r= 0.52, p=0.00, n=37. The bias-corrected FFDI\textsubscript{>25}
agree much better with the station data. Both FFDI metrics illustrate
that fire danger in Victoria increased during the period 1974-2010.

\begin{figure}[H]
\noindent \begin{centering}
\includegraphics[scale=0.5]{Chapter_5/Figures/FFDI/gray_pdfs/FFDI_pdfs}
\par\end{centering}

\caption[Probability density functions of original and bias-corrected Forest
Fire Danger Index (FFDI) results for Victoria, Australia during the
period 1974-2010]{Probability density functions of original and bias-corrected Forest
Fire Danger Index (FFDI) results for Victoria, Australia during the
period 1974-2010. Climate data obtained from the Tweentieth Century
Reanalysis Project (20CR) and weather stations in Victoria (WS). 20CR(bc)
refers to bias-corrected reanalysis data. \label{fig:Probability density functions of orginal and bias-corrected Forest Fire Danger Index results for Victoria, Australia during the period 1974-2010}}
\end{figure}

\begin{figure}[H]
\noindent \begin{centering}
\includegraphics[scale=0.5]{Chapter_5/Figures/FFDI/FFDI_cumulative_1974_2010}
\par\end{centering}

\caption[Cumulative Forest Fire Danger Index (FFDI\protect\textsubscript{cum})
time series during December-January-February (DJF) in Victoria, Australia
for the period 1974-2010]{Cumulative Forest Fire Danger Index (FFDI\protect\textsubscript{cum})
time series during December-January-February (DJF) in Victoria, Australia
for the period 1974-2010. Climate data obtained from the Tweentieth
Century Reanalysis Project (20CR) and weather stations in Victoria
(WS). 20CR(bc) refers to bias-corrected reanalysis data. The correlation
coefficients between weather station and reanalysis data are a) r=0.47,
p=0.00, n=37, and b) r=0.45, p=0.06, n=37. Fire danger displays an
increasing trend in Victoria during the period 1974-2010. \label{fig: Cumulative Forest Fire Danger Index time series during December-January-February in Victoria, Australia for the period 1974-2010} }
\end{figure}


\begin{figure}[H]
\noindent \begin{centering}
\includegraphics[scale=0.5]{Chapter_5/Figures/FFDI/FFDI_g25_1974_2010}
\par\end{centering}

\noindent \centering{}\caption{Number of days with Forest Fire Danger Index greater than 25 (FFDI>25)
time series during December-January-February (DJF) in Victoria, Australia
for the period 1974-2010. Climate data obtained from the Tweentieth
Century Reanalysis Project (20CR) and weather stations in Victoria
(WS). 20CR(bc) refers to bias-corrected reanalysis data. The correlation
coeficients between weather station and reanalysis data are a) r=0.33,
p=0.05, n=37, and b) r= 0.52, p=0.00, n=37. Fire danger displays an
increasing trend in Victoria during the period 1974-2010. \label{fig: Number of days with Forest Fire Danger Index greater than 25 time series during December-January-February in Victoria, Australia for the period 1974-2010}}
\end{figure}


Figures \ref{fig:Probability density functions of original and bias-corrected variables in Victoria, Australia 1974-2010}
and \ref{fig:Time series of original and bias-corrected variables in Victoria, Australia 1974-2010}
show probability density functions (PDFs) of original and bias-corrected
data for Victoria during the period 1974-2010. For every variable
bias-corrected, panel a) displays original data and panel b) the result
with the adjustment. The figures also show the deviations from the
mean and the correlation coeffcients for the PDFs and time series
respectively. On the other hand, Figures \ref{fig:Precipitation probability density functions for Victoria, Australia for the period 1974-2010}
and \ref{fig:Precipitation time series during December-January-February in Victoria, Australia for the period 1974-2010}
illustrate the precipitation PDFs and time series. This variable was
not bias-corrected.

A final step of the FFDI calculation process was to extend the results
using the reanalysis data. The bias-correction process allowed consideration of the period 1920-2010. Figure \ref{fig:Bias-corrected Forest Fire Danger Index (FFDI) time series during December-January-February in Victoria, Australia for the period 1920-2010}
show the results of this computation. The figure includes a FFDI time
series for cumulative FFDI and number of severe days using WS data (period 19740-2010). The time series correlation
coefficients\textemdash comparing the period 1974-2010\textemdash are
a) r=0.47 p=0.00, n=37 and b) r=0.32, p=0.06, n=37.

Figure \ref{fig:Forest Fire Danger Index (FFDI) variables time series during December-January-February in Victoria, Australia for the period 1920-2010}
displays time series of fire weather variables for the period 1920-2010.
The graph includes the time series using original and bias-corrected
reanalysis data. Panel a) shows a clear increasing trend for temperature.
Panels b) and c) depict relative humidity and wind speed (respectively)
with no evident trend. Panel d) shows that precipitation in Victoria
experienced a decline from 1920 to 1980. However, the subsequent decades
display an increasing trend. 

\begin{figure}[H]
\noindent \begin{centering}
\includegraphics[scale=0.7]{Chapter_5/Figures/FFDI_variables/PDFs_FW_variables_Victoria}
\par\end{centering}

\caption[Probability density functions of original and bias-corrected variables
in Victoria, Australia 1974-2010]{Probability density functions of original and bias-corrected variables
in Victoria, Australia 1974-2010. \label{fig:Probability density functions of original and bias-corrected variables in Victoria, Australia 1974-2010} }


\end{figure}


\begin{figure}[H]
\noindent \begin{centering}
\includegraphics[scale=0.7]{Chapter_5/Figures/FFDI_variables/TIME_SERIES_FW_variables_Victoria}
\par\end{centering}

\caption[Time series of original and bias-corrected variables in Victoria,
Australia 1974-2010]{Time series of original and bias-corrected variables in Victoria,
Australia 1974-2010. \label{fig:Time series of original and bias-corrected variables in Victoria, Australia 1974-2010} }


\end{figure}


\begin{figure}[H]
\noindent \begin{centering}
\includegraphics[scale=0.6]{Chapter_5/Figures/FFDI_variables/P/gray_pdfs/P_pdf}
\par\end{centering}

\caption[Precipitation probability density functions for Victoria, Australia
for the period 1974-2010]{Precipitation probability density functions for Victoria, Australia
for the period 1974-2010. This variable was not bias-corrected. Climate
data obtained from the Tweentieth Century Reanalysis Project (20CR)
and weather stations in Victoria (WS). \label{fig:Precipitation probability density functions for Victoria, Australia for the period 1974-2010}}
\end{figure}


\begin{figure}[H]
\noindent \begin{centering}
\includegraphics[scale=0.6]{Chapter_5/Figures/FFDI_variables/P/gray_time_series/P_time_series}
\par\end{centering}

\caption[Precipitation time series during December-January-February (DJF) in
Victoria, Australia for the period 1974-2010]{Precipitation time series during December-January-February (DJF) in
Victoria, Australia for the period 1974-2010. Climate data obtained
from the Tweentieth Century Reanalysis Project (20CR) and weather
stations in Victoria (WS). The pearson correlation coeficients between
weather station and reanalysis data are r=-0.12 and p=0.49. \label{fig:Precipitation time series during December-January-February in Victoria, Australia for the period 1974-2010}}

\end{figure}


\begin{figure}[H]
\noindent \begin{centering}
\includegraphics[scale=0.6]{Chapter_5/Figures/FFDI/FFDI_bc_1920_2010}
\par\end{centering}

\caption[Bias-corrected Forest Fire Danger Index (FFDI) time series during
December-January-February (DJF) in Victoria, Australia for the period
1920-2010]{Bias-corrected Forest Fire Danger Index (FFDI) time series during
December-January-February (DJF) in Victoria, Australia for the period
1920-2010. Panel a) shows cummulative FFDI and panel b) the number
of days with FFDI greater than 25. Climate data obtained from the
Twentieth Century Reanalysis Project (20CR) and weather stations
in Victoria (WS). 20CR(bc) refers to bias-corrected reanalysis data.
Black dashed and red lines show the results using bias-corrected reanalysis
(20CR(bc)) and weather station (WS) data respectively. The correlation
coefficients between reanalysis and weather stations data for the period
1974-2010 are a) r=0.47 p=0.00, n=37 and b) r=0.32, p=0.06, n=37.
Observations and reanalysis data show an increasing fire danger trend
in Victoria. \label{fig:Bias-corrected Forest Fire Danger Index (FFDI) time series during December-January-February in Victoria, Australia for the period 1920-2010}}
\end{figure}


\begin{figure}[H]
\noindent \begin{centering}
\includegraphics[scale=0.8]{Chapter_5/Figures/FFDI_variables/1920-2010/FFDI_variables_time_series}
\par\end{centering}

\caption[Forest Fire Danger Index (FFDI) variables time series during December-January-February
(DJF) in Victoria, Australia for the period 1920-2010]{Forest Fire Danger Index (FFDI) variables time series during December-January-February
(DJF) in Victoria, Australia for the period 1920-2010. Climate data
obtained from the Tweentieth Century Reanalysis Project (20CR). 20CR(bc)
refers to bias-corrected reanalysis data. Temperature and precipitation
show an increasing and decreasing trend respectively. Relative humidity
and wind speed do not display any trend. \label{fig:Forest Fire Danger Index (FFDI) variables time series during December-January-February in Victoria, Australia for the period 1920-2010}}


\end{figure}



\subsubsection{Victorian Seasonal Bushfire Index computation}

The bias-corrected reanalysis data was also used to compute the VSBI.
The VSBI is an alternative fire weather metric for Victoria. The VSBI
equation combines local climate variables and ENSO indicators (see
section \ref{sub:Formulation-VSBI}). Figure \ref{fig:Victorian Seasonal Bushfire Index time series during December-January-February in Victoria, Australia for the period 1920-2010}
shows a time series of this index during DJF for the period 1920-2010.
Fire weather represented by the VSBI has a strong agreement between
observations and reanalysis data (r=0.69, p<0.05).

\begin{figure}[h]
\noindent \begin{centering}
\includegraphics{Chapter_5/Figures/VSBI/VSBI_20CR_1920_2010}
\par\end{centering}

\caption[Victorian Seasonal Bushfire Index (VSBI) time series during December-January-February
(DJF) in Victoria, Australia for the period 1920-2010]{Victorian Seasonal Bushfire Index (VSBI) time series during December-January-February
(DJF) in Victoria, Australia for the period 1920-2010. The computation
used bias-corrected Twentieth Century Reanalysis data (20CR(bc)) (black
dashed line) and weather station data (red line). The correlation
coefficients between reanalysis and weather station data for the period
1974-2010 are r=0.69 and p<0.05. Observations and reanalysis data
display an incresing fire danger trend in Victoria. \label{fig:Victorian Seasonal Bushfire Index time series during December-January-February in Victoria, Australia for the period 1920-2010} }
\end{figure}


Figure \ref{fig:Victorian Seasonal Bushfire Index variables time series during December-January-February in Victoria, Australia for the period 1920-2010}
displays a time series of VSBI variables during DJF for the period
1920-2010. The figure shows four panels. Panel a) indicates that temperature
over Victoria has increased over the 91 years period. Panel b) does
not illustrate long-term trends for relative humidity. Panel c) shows
the sea surface temperature variability on the NINO1+2 region. The
reanalysis data captures the strong \textquotedblleft El Ni\~no\textquotedblright{}
events of 1982 and 1997. In panel d), precipitation in Victoria does
not display a trend. 

\begin{figure}[h]
\noindent \begin{centering}
\includegraphics[scale=0.7]{Chapter_5/Figures/VSBI/VSBI_variables_1920_2010}
\par\end{centering}

\caption[Victorian Seasonal Bushfire Index (VSBI) variables time series during
December-January-February (DJF) in Victoria, Australia for the period
1920-2010]{Victorian Seasonal Bushfire Index (VSBI) variables time series during
December-January-February (DJF) in Victoria, Australia for the period
1920-2010. The time series correspond to normalized spatial averages
of temperature (T), relative humidity, sea surface temperature (SST)
and precipitation (P) over Victoria, Australia during December-January-February
(DJF). The atmospheric variables correspond to bias-corrected Twentieth
Century Reanalysis (20CR) data. The SST time series use the the HadISST
dataset in the NINO1+2 region. Only temperature shows a clear increasing
trend. \label{fig:Victorian Seasonal Bushfire Index variables time series during December-January-February in Victoria, Australia for the period 1920-2010}}


\end{figure}



\subsection{Seasonal forecasting skill}

\LyXZeroWidthSpace The correlation analysis shows climate-bushfire
relationships in spring (SON-SON), summer (DJF-DJF) and from spring
to summer (SON-DJF) in Victoria. The analysed periods were 1974-2010,
1961-2010 and 1920-2010. These periods represent the years with available
weather observations, bushfire records, and reanalysis data. The correlations
used bias-corrected 20CR reanalysis and weather station data. The
results in coloured circles represent correlations with a 5\% statistical
significance level. The size of the circles represents the strength
of the correlations. 


\subsubsection{Spring climate-bushfire relationships (SON-SON)}

Figures \ref{fig:Pearson correlation coeficients for climate-bushfire relationships for September-October-November in Victoria during the period 1973-2009},
\ref{fig:Pearson correlation coeficients for climate-bushfire relationships for September-October-November in Victoria during the period 1960-2009}
and \ref{fig:Pearson correlation coeficients for climate-bushfire relationships for September-October-November in Victoria during the period 1919-2009}
show the linear correlation coefficients of the analysed variables
in spring (SON-SON). The results indicate that during spring the VSBI
and FFDI seasonal metrics are highly correlated. The relationship
is more consistent for the first period of analysis (1974-2010). In
this period, VSBI shows strong and statistically significant relationships
with the two FFDI metrics (FFDI\textsubscript{cum} and FFDI\textsubscript{>25}).
However, only the FFDI\textsubscript{cum} metric still shows this
relationship further back in time\textemdash using reanalysis data\textemdash .
Additionally, the correlations between the FFDI metrics also decrease
over time. These correlations start with strong relationships, with
r=0.9 and r=0.78, using WS and 20CR data respectively. However, they
decrease to r=0.37 and r=0.36 for the second and third period respectively.
Overall, VSBI and FFDI\textsubscript{cum} have a stronger correlation
than VSBI and FFDI\textsubscript{>25}. 

The climate variable that most influences fire weather is relative
humidity. The correlations between this variable and VSBI fluctuate
from -0.79 to -0.95 for the three periods. For FFDI\textsubscript{cum},
the range of correlations is -0.84 to -0.97. Every result was statistically
significant for these two metrics. The FFDI\textsubscript{>25} metric
shows, for the first period, correlations coeffcients of -0.81 and
-0.73 with WS and 20CR data respectively. However, this metric shows
no statistically significant results using reanalysis data for the
last two periods. 

Climate indices show a moderate relationship with fire weather metrics.
Additionally, the \textquotedblleft El Ni\~no\textquotedblright{} component
of the index (SST1.2) has the least influence over the VSBI in comparison
to local climate variables\textemdash as expected\textemdash . The
correlations between VSBI and SST1.2 are statistically significant
for the three periods. They fluctuate from 0.51 to 0.62. On the other
hand, the correlations with modes of climate variability and FFDI
metrics show similar results for ENSO and IOD indices. However, the
Dipole Mode Index (DMI) presents slightly higher correlations. These
correlations are moderate for the first period. Further back in time,
they become weak and statistically not significant. 

\begin{figure}[h]
\begin{lyxlist}{00.00.0000}
\item [{\includegraphics[scale=0.75]{Chapter_5/Figures/Correlations/1974-2010/20CR/SON-SON/SON-SON}}]~
\end{lyxlist}
\caption[Pearson correlation coeficients for climate-bushfire relationships
for September-October-November (SON) in Victoria during the period
1973-2009]{Pearson correlation coeficients for climate-bushfire relationships
for September-October-November (SON) in Victoria during the period
1973-2009. Data source: Twentieth Century Reanalysis (20CR), Hadley
Centre Sea Ice and Sea Surface Temperature data set (HadISST), and
weather station data in Victoria (WS). \label{fig:Pearson correlation coeficients for climate-bushfire relationships for September-October-November in Victoria during the period 1973-2009}}


\end{figure}


\begin{figure}[h]
\noindent \begin{centering}
\includegraphics[scale=0.8]{Chapter_5/Figures/Correlations/1961-2010/20CR/SON-SON/SON-SON}
\par\end{centering}

\caption[Pearson correlation coeficients for climate-bushfire relationships
for September-October-November (SON) in Victoria during the period
1960-2009]{Pearson correlation coeficients for climate-bushfire relationships
for September-October-November (SON) in Victoria during the period
1960-2009. Data source: Twentieth Century Reanalysis (20CR), Hadley
Centre Sea Ice and Sea Surface Temperature data set (HadISST), and
weather station data in Victoria (WS). \label{fig:Pearson correlation coeficients for climate-bushfire relationships for September-October-November in Victoria during the period 1960-2009}}
\end{figure}


\begin{figure}[h]
\noindent \begin{centering}
\includegraphics[scale=0.75]{Chapter_5/Figures/Correlations/1920-2010/20CR/SON-SON/SON-SON}
\par\end{centering}

\caption[Pearson correlation coeficients for climate-bushfire relationships
for September-October-November (SON) in Victoria during the period
1919-2009]{Pearson correlation coeficients for climate-bushfire relationships
for September-October-November (SON) in Victoria during the period
1919-2009. Data source: Twentieth Century Reanalysis (20CR), Hadley
Centre Sea Ice and Sea Surface Temperature data set (HadISST), and
weather station data in Victoria (WS). \label{fig:Pearson correlation coeficients for climate-bushfire relationships for September-October-November in Victoria during the period 1919-2009}}


\end{figure}



\subsubsection{Summer climate-bushfire relationships (DJF-DJF)}

Figures \ref{fig:Pearson correlation coeficients for climate-bushfire relationships for December-January-February in Victoria during the period 1974-2010},
\ref{fig:Pearson correlation coeficients for climate-bushfire relationships for December-January-February in Victoria during the period 1961-2010}
and \ref{fig:Pearson correlation coeficients for climate-bushfire relationships for December-January-February in Victoria during the period 1920-2010}
show the linear correlation results for summer (DJF-DJF). During this
season, the relationship between VSBI and FFDI metrics decreases in
comparison to spring. For example, correlations between these metrics
using WS data in SON ranged from 0.78 to 0.92 for the first period
1973-2010. However, the correlations decrease to 0.46 and 0.69 analysing
the same data during summer. The correlations in summer agree with
the results in spring showing a strong agreement between VSBI and
FFDI\textsubscript{cum}.

There are some differences on which fire weather metric correlates
better with fire activity. For this season, WS data show that FFDI
metrics have a stronger correlation with fire activity than the VSBI.
Using observations, the correlations with FFDI metrics and fire activity
are higher with Risk Frontiers database (r=0.74 and r=0.72 with FFDI\textsubscript{cum}
and FFDI\textsubscript{>25} respectively). The correlation
with Victorian Government's database are also high, but slightly (r=0.64 with both
metrics). In contrast, reanalysis data suggests that the VSBI presents
better correlations with fire activity for the two periods (1974-2010
and 1961-2010).

There are also disagreements between the analysis using WS and 20CR
data to establish which climate variable has the strongest influence
during this season. WS data indicates that relative humidity is the
main climate driver of fire weather and activity. The result is consistent
in spring and summer. In contrast, reanalysis data suggests temperature
to have a greater influence in summer.

Modes of climate variability have less influence over fire weather
in summer. Unexpectedly, correlations betwen climate indices and fire activity 
show to be stronger than correlation between these indices and fire weaher. This result is 
unexpected because fire activity depends on factors beyond weather. \textquotedblleft El
Ni\~no\textquotedblright{} indices are moderately correlated with fire
weather. On the other hand, there is no correlation between fire metrics
with the DMI. This lack of influence is also an expected result as
the IOD's influence weakens after spring. 

\begin{figure}[h]
\noindent \begin{centering}
\includegraphics[scale=0.75]{Chapter_5/Figures/Correlations/1974-2010/20CR/DJF-DJF/DJF-DJF}
\par\end{centering}

\caption[Pearson correlation coefficients for climate-bushfire relationships
for December-January-February (DJF) in Victoria during the period
1974-2010]{Pearson correlation coefficients for climate-bushfire relationships
for December-January-February (DJF) in Victoria during the period
1974-2010. Data source: Twentieth Century Reanalysis (20CR), Hadley
Centre Sea Ice and Sea Surface Temperature data set (HadISST), and
weather stations data in Victoria.\label{fig:Pearson correlation coeficients for climate-bushfire relationships for December-January-February in Victoria during the period 1974-2010}}
\end{figure}


\begin{figure}[h]
\noindent \begin{centering}
\includegraphics[scale=0.75]{Chapter_5/Figures/Correlations/1961-2010/20CR/DJF-DJF/DJF-DJF}
\par\end{centering}

\caption[Pearson correlation coeficients for climate-bushfire relationships
for December-January-February (DJF) in Victoria during the period
1961-2010]{Pearson correlation coeficients for climate-bushfire relationships
for December-January-February (DJF) in Victoria during the period
1961-2010. Data source: Twentieth Century Reanalysis (20CR), Hadley
Centre Sea Ice and Sea Surface Temperature data set (HadISST), and
weather stations data in Victoria.\label{fig:Pearson correlation coeficients for climate-bushfire relationships for December-January-February in Victoria during the period 1961-2010}}
\end{figure}


\begin{figure}[h]
\noindent \begin{centering}
\includegraphics[scale=0.8]{Chapter_5/Figures/Correlations/1920-2010/20CR/DJF-DJF/DJF-DJF}
\par\end{centering}

\caption[Pearson correlation coeficients for climate-bushfire relationships
for December-January-February (DJF) in Victoria during the period
1920-2010]{Pearson correlation coeficients for climate-bushfire relationships
for December-January-February (DJF) in Victoria during the period
1920-2010. Data source: Twentieth Century Reanalysis (20CR), Hadley
Centre Sea Ice and Sea Surface Temperature data set (HadISST), and
weather stations data in Victoria.\label{fig:Pearson correlation coeficients for climate-bushfire relationships for December-January-February in Victoria during the period 1920-2010}}
\end{figure}



\subsubsection{Spring-summer bushfire forecasting skill (SON-DJF)}

Figures \ref{fig:Pearson correlation coeficients for climate-bushfire relationships from September-October-November to December-January-February in Victoria, Australia during the period 1973-2010 (part 1)},
\ref{fig:Pearson correlation coeficients for climate-bushfire relationships from September-October-November to December-January-February in Victoria, Australia during the period 1973-2010 (part 2)},
\ref{fig:Pearson correlation coeficients for climate-bushfire relationships from September-October-November to December-January-February in Victoria, Australia during the period 1961-2010}
and \ref{fig:Pearson correlation coeficients for climate-bushfire relationships from September-October-November to December-January-February in Victoria, Australia during the period 1919-2010}
display the linear correlations from spring to summer (SON-DJF). Figure
\ref{fig:Pearson correlation coeficients for climate-bushfire relationships from September-October-November to December-January-February in Victoria, Australia during the period 1973-2010 (part 1)}
and \ref{fig:Pearson correlation coeficients for climate-bushfire relationships from September-October-November to December-January-February in Victoria, Australia during the period 1973-2010 (part 2)}
show two panels with correlations using reanalysis (panel a) and weather
station data (panel b) during the period 1973-2010. Figure \ref{fig:Pearson correlation coeficients for climate-bushfire relationships from September-October-November to December-January-February in Victoria, Australia during the period 1973-2010 (part 1)}
shows the results correlating climate variables, fire weather and
bushfire events. On the other hand, in Figure \ref{fig:Pearson correlation coeficients for climate-bushfire relationships from September-October-November to December-January-February in Victoria, Australia during the period 1973-2010 (part 1)}
the correlations use climate indices instead of climate variables.
Figures \ref{fig:Pearson correlation coeficients for climate-bushfire relationships from September-October-November to December-January-February in Victoria, Australia during the period 1961-2010}
and \ref{fig:Pearson correlation coeficients for climate-bushfire relationships from September-October-November to December-January-February in Victoria, Australia during the period 1919-2010}
show the results for the other two periods using reanalysis data only.
Panels a) and b) in these figures also compare the results using climate
variables and remote drivers indices in the correlations.

The VSBI forecasting skill show inconsistencies between weather station
and reanalysis data. Using WS data, the VSBI shows a strong skill
to forecast fire weather. The correlations with FFDI metrics are 0.74
and 0.71 for FFDI\textsubscript{cum} and FFDI\textsubscript{>25}
respectively. Using weather observations, the correlations are weaker for
fire activity which depends on not only weather. Therefore, this result
is consistent. The correlations between VSBI and bushfire events are
0.66 and 0.54 using Risk Frontiers (RF) and Victorian's Government
(DELWP) databases respectively. On the other hand, with reanalysis
data the VSBI has a greater skill to forecast fire activity for the periods 1974-2010 and 1960-2010. 
For the last period, the VSBI only shows a statistically significant\textemdash although
weak\textemdash relationship with FFDI\textsubscript{cum}. 

The results suggest that relative humidity is the climate variable
with the strongest forecasting skill. Using observations, the correlations
between spring RH and summer FFDI metrics are -0.71 and -0.68\textemdash with
FFDI\textsubscript{cum} and FFDI\textsubscript{>25} respetively\textemdash .
Additionally, relative humidity has a correlation of -0.64 and -0.58
with RF and DELWP databases repectively. On the other hand, using
reanalysis data the correlations between this variable and fire metrics
deacrease. Although, these data consistently show that this variable
has greater forecasting skill than temperature and precipitation. 

Climate variables also demonstrate predictive skill disagreements
between observations and reanalysis data. The WS data show that climate
variables have a comparable skill to forecast fire weather and bushfire
events. In contrast, reanalysis data indicate a stronger relationship
between climate indices and fire activity. These inconsistencies occur
in the two periods with available fire activity data (1973-2010 and
1960-2010). Using observations, the correlations show a moderate
fire (weather and events) forecasting skill. 

Finally, weather station and reanalysis data agree to show that the
VSBI has greater forecasting skill than individual climate variables
and indices. For example, panels b) in Figures \ref{fig:Pearson correlation coeficients for climate-bushfire relationships from September-October-November to December-January-February in Victoria, Australia during the period 1973-2010 (part 1)}
and \ref{fig:Pearson correlation coeficients for climate-bushfire relationships from September-October-November to December-January-February in Victoria, Australia during the period 1973-2010 (part 2)}
demonstrate\textemdash using observations\textemdash how the predictive
skill of the index is increased with the combination of local climate
variables and the ENSO indicator. The results using reanalysis data
also show this effect in the three periods examined.

\begin{figure}[h]
\noindent \begin{centering}
\includegraphics[scale=0.75]{Chapter_5/Figures/Correlations/1974-2010/20CR/SON-DJF/SON-DJF_part_1}
\par\end{centering}

\caption[Pearson correlation coeficients for climate-bushfire relationships
from September-October-November (SON) to December-January-February
(DJF) in Victoria, Australia during the period 1973-2010]{Pearson correlation coeficients for climate-bushfire relationships
from September-October-November (SON) to December-January-February
(DJF) in Victoria, Australia during the period 1973-2010. Panel a)
and b) show the results for climate variables, fire weather and activity
using reanalysis and weather station data respectively. Data source:
Twentieth Century Reanalysis (20CR), Hadley Centre Sea Ice and Sea
Surface Temperature data set (HadISST), weather stations in Victoria
(WS). \label{fig:Pearson correlation coeficients for climate-bushfire relationships from September-October-November to December-January-February in Victoria, Australia during the period 1973-2010 (part 1)}}
\end{figure}


\begin{figure}[h]
\noindent \begin{centering}
\includegraphics[scale=0.75]{Chapter_5/Figures/Correlations/1974-2010/20CR/SON-DJF/SON-DJF_part_2}
\par\end{centering}

\caption[earson correlation coeficients for climate-bushfire relationships
from September-October-November (SON) to December-January-February
(DJF) in Victoria, Australia during the period 1973-2010]{Pearson correlation coeficients for climate-bushfire relationships
from September-October-November (SON) to December-January-February
(DJF) in Victoria, Australia during the period 1973-2010. Panel a)
and b) show the results for climate indices, fire weather and activity
using reanalysis and weather station data respectively. Data source:
Twentieth Century Reanalysis (20CR), Hadley Centre Sea Ice and Sea
Surface Temperature data set (HadISST), weather stations in Victoria
(WS).\label{fig:Pearson correlation coeficients for climate-bushfire relationships from September-October-November to December-January-February in Victoria, Australia during the period 1973-2010 (part 2)}}
\end{figure}


\begin{figure}[h]
\noindent \begin{centering}
\includegraphics[scale=0.75]{Chapter_5/Figures/Correlations/1961-2010/20CR/SON-DJF/SON-DJF}
\par\end{centering}

\caption[Pearson correlation coeficients for climate-bushfire relationships
from September-October-November (SON) to December-January-February
(DJF) in Victoria, Australia during the period 1961-2010]{Pearson correlation coeficients for climate-bushfire relationships
from September-October-November (SON) to December-January-February
(DJF) in Victoria, Australia during the period 1961-2010. Panel a)
and b) show the results for climate variables and climate indices
respectively. Data source: Twentieth Century Reanalysis (20CR) and
Hadley Centre Sea Ice and Sea Surface Temperature data set (HadISST).\label{fig:Pearson correlation coeficients for climate-bushfire relationships from September-October-November to December-January-February in Victoria, Australia during the period 1961-2010}}
\end{figure}


\begin{figure}[h]
\noindent \begin{centering}
\includegraphics[scale=0.75]{Chapter_5/Figures/Correlations/1920-2010/20CR/SON-DJF/SON-DJF}
\par\end{centering}

\caption[Pearson correlation coeficients for climate-bushfire relationships
from September-October-November (SON) to December-January-February
(DJF) in Victoria, Australia during the period 1919-2010]{Pearson correlation coeficients for climate-bushfire relationships
from September-October-November (SON) to December-January-February
(DJF) in Victoria, Australia during the period 1919-2010. Panel a)
and b) show the results for climate variables and climate indices
respectively. Data source: Twentieth Century Reanalysis (20CR) and
Hadley Centre Sea Ice and Sea Surface Temperature data set (HadISST).\label{fig:Pearson correlation coeficients for climate-bushfire relationships from September-October-November to December-January-February in Victoria, Australia during the period 1919-2010}}
\end{figure}



\section{Discussion}


\subsection{Seasonal fire weather variability \label{sub:Seasonal-fire-weather-variability}}


\subsubsection{Original seasonal FFDI computation (1974-2010)}

The FFDI computed with WS data depict more extreme fire weather for
Victoria than the calculation using the 20CR. This difference is likely
to occur for several reasons. FFDI variables from reanalysis data
do not correspond exactly to surface values. Near-surface reanalysis
data was available at the 1000 hPa level. Calculations using these
data do not exactly represent surface level conditions. Additionally,
the FFDI computed with weather station data use values at 3 pm. Yet,
reanalysis data for FFDI variables is available at 6-hourly intervals.
These intervals limited the availability of data to 4 pm values. Moreover,
there are intrinsic limitations in the reanalysis dataset already
discussed in Chapter 2. However, the fire weather variability could
change if we use a different index. 

An alternative index might show a different fire weather variability.
Yet, the difference is unlikely to be significant. \citet{Dowdy2009}
determined that the FFDI is comparable to the Canadian Fire
Weather Index (FWI). The study compared the two indices for Australian
conditions. The FWI is an internationally recognized fire weather
index \citep{VanWagner1974,VanWagner2005,Dowdy2009a}. Therefore,
the FFDI can be representative of fire weather conditions in Victoria.
Nevertheless, the use of only one realization of the 20CR leaves non-quantified
uncertainties. 

The 20CR reanalysis has 56 model realizations \citep{Compo2011}.
Every realization represents an equally possible state of the atmosphere
\citep{Compo2011}. Therefore, fire weather computed with a mean value
realization has a degree of uncertainty. This uncertainty could be
quantified. Thus, the bias-correction results are only valid for one
possible state of the atmosphere. However, the seasonal fire weather
computations show an increasing trend consistent with an atmosphere
under a global warming process \citep{Stocker2013}. 

The increasing trend in seasonal fire weather computations for Victoria
agrees with the results of other investigations for southeastern Australia
\citep{Lucas2007,Hennessy2005,Hasson2009,Clarke2013}. In fact, climate
change is extending the bushfire season in this region \citep{Clarke2011}.
However, the available databases showed that most bushfires occur
during DJF. Therefore, in this investigation the fire season was defined
as the season December-January-February. The results can change\textemdash although
not significantly- if the bushfire season is extended (e.g. November-December-January-February-March\textemdash ). 


\subsubsection{Bias-correction of fire weather variables (1974-2010)}

Every fire weather variable showed deviations from observations. These
biases occurred due to three reasons. First of all, reanalysis data
does not represents surface values. The available 20CR data comprises
climate fields at the 1000 hPa level. Additionally, the time of the
reanalysis outputs differs from observations\textemdash{} 4 and 3
pm respectively\textemdash . The coarse reanalysis resolution and
the scarce number of WS with avaible data also bring biases. On the
other hand, some variables capture the observed climate variability
better than others (e.g. temperature and relative humidity). 

Although, the bias-correction process yielded satisfactory results,
weather station data was not homogenized. This is a limitation that
is particularly evident for wind speed. The seasonal values for this
variable show an abrupt increase after the year 2000. The change in
the technology to measure the variable might cause the inhomogeneities.
On the other hand, linear scaling techniques did not correct biases
in precipitation. 

The 20CR data does not accurately represents precipitation in Victoria.
The linear scaling bias-correction techniques did not yield better
results for this variable. Therefore, the FFDI computation did not
use bias-corrected data for precipitation. However, the FFDI is less
sensitive to changes in this variable \citep{Dowdy2009a}. This weaker
sensitivity might occur because the FFDI equation does not explicitly
includes precipitation. This variable is part of the drought factor.
This factor does not include precipitation explicitly either. The
drought factor computation requires the calculation of moisture deficits.
In this research, the use of the Keet-Byram Drought Index (KBDI) \citep{KeetchJJ}
yields these deficits.


\subsubsection{Bias-corrected seasonal FFDI computation (1974-2010)}

After the bias-correction, the magnitude of FFDI results using reanalysis
data is comparable to the computation with observations for the period
1974-2010. During this period, the two types of data show that Victoria
experienced a steep increase in fire danger. However, the FFDI computed
with bias-corrected reanalysis data slightly over-estimates seasonal
fire weather. On the other hand, FFDI\textsubscript{cum} seems to
be a better metric of seasonal fire weather than FFDI\textsubscript{>25}.
The results with FFDI\textsubscript{cum} show a higher correlation
between bushfire danger and activity. The result is consistent using
weather station and alternative reanalysis data (see Chapter 6). This
result is also similar using bias-corrected data. 


\subsubsection{Extended seasonal FFDI computation (1920-2010)}

Fire weather analysed using the 20CR data shows that Victoria experienced
an increasing fire danger trend during the period 1920-2010. However,
there is greater uncertainty in the accuracy of the results for this
period. Fire weather variables could show even larger biases before
the period 1974-2010. The discrepancies might be considerable for
variables like wind speed because of the incorporation of new measuring
technologies. Additionally, after the bias-correction the seasonal
FFDI underestimates fire weather during the period 1974-2010. This
result is not consistent with the overestimation yielded for the period
1974-2010. There is no weather station data to compare the fields
for the period 1920-1973. Therefore, an alternative approach is to
use an independent reanalysis product like ERA-20C \citep{Stickler2014}.
This analysis is shown in Chapter 6.


\subsubsection{Computation of the Victorian Seasonal Bushfire Index}

The VSBI agrees with the FFDI metrics showing an increasing fire weather
trend during the period 1920-2010. Additionally, this metric demonstrates
the strongest correlation between observations and reanalysis data
(r=0.69 and p<0.05). The VSBI time series shows that this fire weather
metric is particularly sensitive to the sea surface temperature parameter.
Figure \ref{fig:Victorian Seasonal Bushfire Index time series during December-January-February in Victoria, Australia for the period 1920-2010}
depicts how the VSBI peaks during the two strong ``El Ni\~no'' events
in the years 1982-1983 and 1997-1998. Figure \ref{fig:Victorian Seasonal Bushfire Index variables time series during December-January-February in Victoria, Australia for the period 1920-2010}
displays how these peaks are clearly linked to the sea surface temperature
variability. The next section discusses weather the VSBI is a better
metric of seasonal fire danger than FFDI metrics and its skill to
forecast fire activity. 


\subsection{Seasonal forecasting skill}

This section discusses the results of the linear correlation analyses.
The section focuses on the influence of climate drivers over fire
weather and bushfire events in spring and summer. Additionally, it
also comments on the seasonal predictability skill. Further discussion
about these topics takes place in Chapter 6.


\subsubsection{Fire weather drivers}

The relationship between variables in spring shows a moderate influence
of the IOD. These results agree with the findings of \citet{Cai2009}.
Positive IOD events drive hot and dry weather conditions during spring
in Victoria. These environmental conditions dry fuel loads. Dried
fuel loads makes the summer season more prone to bushfires. The IOD
exerts a stronger influence than ENSO in this season. This result
also agrees with the findings of \citet{Cai2009}. However, there
is a strong correlation between the DMI and ENSO indices. Therefore,
the IOD might not be independent of ENSO as other studies suggest
\citep{Dommenget2002}. In fact, ENSO indices show how ``El Ni\~no''
events also drive Victorian climate in spring.

Among the ENSO indices, NINO3.4 showed the strongest relationships
with climate variables. This strong correlation reflects how the oceanic
component of ENSO influences fire weather. In the previous chapter,
seasonal anomalies in this region showed statistically significant
results. These anomalies have a link with bushfire occurrences in
Victoria. These patterns justified the decision to include the SST
region in the VSBI formulation. In contrast, SAM does not exert a
significant influence preconditioning bushfire seasons in spring.
Appendix C shows the correlations for this variable.

The strong influence of SAM over rainfall in Victoria is well known
\citep{Risbey2009b,Ho2011}. Therefore, the lack of influence of SAM
over fire weather is an unexpected result. Probably SAM has a stronger
influence in other seasons (e.g. winter). On the other hand, the use
of different climatologies could explain the weak influence of SAM
(see Appendix C). The climatology used for bushfire variables was
1961-2011. On the other hand, the climatology for SAM was 1979-2000.
Although the correlation results for climate indices vary from spring
to summer.

Overall, the correlations between climate indices and bushfire metrics
in DJF are not strong. Thus, these findings contrast with the results
in the previous season. The influence of ENSO and IOD is greater in
spring than summer. These modes of variability have more influence
building-up fuels loads in spring, rather than exacerbating bushfire
danger in summer. However, ENSO and IOD do not show much skill to
forecast fire weather on their own.


\subsubsection{Seasonal fire weather predictability}

Relative humidity and temperature in spring are strong predictors
of bushfire metrics in summer. Moreover, relative humidity is the
variable that most influences fire weather in Victoria. This result
is consistent using weather station and reanalysis data. Additionally,
the correlations for relative humidity show similar results in the
tests performed for SON, DJF and from SON to
DJF. This result agrees with the findings of \citet{Harris2013}
using vapor pressure as predictor. However, it is worth acknowledging
again that weather stations data were not homogenized. Additionally,
precipitation from reanalysis data was not bias-corrected. Finally,
it is also important to contrast this result with the skill of the
VSBI.

Computations with WS data suggest that the VSBI could be useful to
forecast fire weather\textemdash and even fire activity\textemdash in
Victoria. The results using these data yielded the expected outcomes.
The VSBI computed with observations demonstrated a higher forecasting
skill than local climate variables and climate indices on their own.
Indeed, the incorporation of the ENSO indicator in the VSBI equation
shows how it increases the forecasting skill of the index. On the
other hand, the VSBI demonstrated a higher forecasting skill for fire
weather than bushfires. This is also an expected outcome since fire
activity not only depends on weather (e.g. ignitions sources). The
index also proved to yield similar results to FFDI seasonal metrics,
with the advantage to be simple to compute. However, the use of the
20CR reanalysis in the computation of the index did not confirmed
the results. It could be the case that the resolution of the reanalysis
data is not suitable for regional long-term analysis. Therefore, the
next chapter uses an alternative reanalysis dataset to perform further
tests to the index. 


\section{Summary}

This investigation explored the fire weather variability in Victoria.
Seasonal metrics of the McArthur Forest Fire Danger Index quantified
bushfire danger using meteorological data. This chapter also proposed
an alternative fire weather index: the \textquotedblleft Victorian
Seasonal Bushfire Index\textquotedblright{} (VSBI). The VSBI represents
fire weather associating local climate variables with an El Ni\~no-Southern
Oscillation indicator. The computation of the two indices used weather
station and bias-corrected reanalysis data. Linear scaling techniques
were used to bias-correct these data. Additionally, the chapter explores
the relationship between bushfire activity, fire weather, climate
modes of variability and meteorological variables. The relationships
were analysed computing linear correlations for several variables
in spring, summer and from spring to summer. These analyses revealed
the most important fire weather drivers for each season and the forecasting
skill of the VSBI.

Computing fire weather using observations demonstrates that Victoria
experienced an increasing trend in fire danger during the period 1974-2010.
This tendency is consistent using different fire weather metrics.
Reanalysis data also shows the increasing trend in fire danger. In
fact, the 20CR data shows that the fire danger increase can be traced
back to the early years of the 20\textsuperscript{th} century. On
the other hand, the VSBI demonstrated to be comparable with the FFDI.
Moreover, the VSBI also shows that incorporating an ENSO indicador
in a fire weather index increases the skill to forecast extreme fire
seasons. Observations and model data agree that relative humidity
is the climate variable that has the greatest influence over fire
weather\textemdash and fire activity\textemdash in Victoria. Additionally,
modes of climate variability play a moderate role in driving fire
weather in this region. ENSO and IOD are the most important drivers,
and their influence is greater in spring than summer. However, these
results require further examination using a different reanalysis product.

The next chapter aims to test the sensitivity of Chapter's 4 and 5
results. The bushfire and heatwave weather patterns in Victoria are
reproduced using an alternative fire activity database and heatwave
definition. Additionally, the fire weather variability and forecasting
skill in this region are analysed using a second reanalysis dataset. 
